%----------------------------------------------------------------------------------------
%	PACKAGES AND OTHER DOCUMENT CONFIGURATIONS
%----------------------------------------------------------------------------------------

\documentclass[
11pt, % The default document foCnt size, options: 10pt, 11pt, 12pt
oneside, % Two side (alternating margins) for binding by default, uncomment to switch to one side
ngerman, % ngerman for German
singlespacing, % Single line spacing, alternatives: onehalfspacing or doublespacing
%draft, % Uncomment to enable draft mode (no pictures, no links, overfull hboxes indicated)
%nolistspacing, % If the document is onehalfspacing or doublespacing, uncomment this to set spacing in lists to single
%liststotoc, % Uncomment to add the list of figures/tables/etc to the table of contents
%toctotoc, % Uncomment to add the main table of contents to the table of contents
%parskip, % Uncomment to add space between paragraphs
%nohyperref, % Uncomment to not load the hyperref package
headsepline, % Uncomment to get a line under the header
%chapterinoneline, % Uncomment to place the chapter title next to the number on one line
%consistentlayout, % Uncomment to change the layout of the declaration, abstract and acknowledgements pages to match the default layout
]{MasterThesis} % The class file specifying the document structure

\usepackage[utf8]{inputenc} % Required for inputting international characters
\usepackage[T1]{fontenc} % Output font encoding for international characters

\usepackage{mathpazo} % Use the Palatino font by default

\usepackage[backend=bibtex,natbib=true]{biblatex} % Use the bibtex backend with the authoryear citation style (which resembles APA) removed: style=authoryear,

\addbibresource{bibliography.bib} % The filename of the bibliography

\usepackage[autostyle=true]{csquotes} % Required to generate language-dependent quotes in the bibliography

\usepackage{caption}
\usepackage{framed} % rahmen um abbildungen
\usepackage{enumerate}

\usepackage{float} % um bilder fest zu positionieren mittels H

% Für Logo auf Titelseite %
\usepackage[absolute]{textpos}	% Bilder absolut positionieren. 
\setlength{\TPHorizModule}{1mm}
\setlength{\TPVertModule}{\TPHorizModule}
\textblockorigin{0mm}{0mm}
% %%%%%%%%%%%%%%%%%%%%%%% %

\usepackage{pdfcomment} % für pdf kommentare

% Mathepakete
\usepackage{amsfonts}
\usepackage{amsmath}

% For java code snippets 

\usepackage{color}

\definecolor{pblue}{rgb}{0.13,0.13,1}
\definecolor{pgreen}{rgb}{0,0.5,0}
\definecolor{pred}{rgb}{0.9,0,0}
\definecolor{pgrey}{rgb}{0.46,0.45,0.48}

\usepackage{listings}
\lstset{language=Java,
  showspaces=false,
  showtabs=false,
  breaklines=true,
  numbers=left,
  showstringspaces=false,
  breakatwhitespace=true,
  commentstyle=\color{pgreen},
  keywordstyle=\color{pblue},
  stringstyle=\color{pred},
  basicstyle=\ttfamily,
  moredelim=[il][\textcolor{pgrey}]{\$\$},
  moredelim=[is][\textcolor{pgrey}]{\%\%}{\%\%}
}

%----------------------------------------------------------------------------------------
%	MARGIN SETTINGS
%----------------------------------------------------------------------------------------

\geometry{
	paper=a4paper, % Change to letterpaper for US letter
	inner=2.5cm, % Inner margin
	outer=3.8cm, % Outer margin
	bindingoffset=.5cm, % Binding offset
	top=1.5cm, % Top margin
	bottom=1.5cm, % Bottom margin
	%showframe, % Uncomment to show how the type block is set on the page
}

%----------------------------------------------------------------------------------------
%	THESIS INFORMATION
%----------------------------------------------------------------------------------------
%
\thesistitle{Einsatz und Vergleich verschiedener Blockchain-Technologien am Beispiel einer Glücksspielanwendung} % Your thesis title, this is used in the title and abstract, print it elsewhere with \ttitle
\supervisor{Prof. Dr. rer. nat. Dr.-Ing. Georg \textsc{Hoever}} % Your supervisor's name, this is used in the title page, print it elsewhere with \supname
\examiner{} % Your examiner's name, this is not currently used anywhere in the template, print it elsewhere with \examname
\degree{Information System Engineering} % Your degree name, this is used in the title page and abstract, print it elsewhere with \degreename
\author{Dany \textsc{Brossel}} % Your name, this is used in the title page and abstract, print it elsewhere with \authorname
\addresses{} % Your address, this is not currently used anywhere in the template, print it elsewhere with \addressname

\subject{IT} % Your subject area, this is not currently used anywhere in the template, print it elsewhere with \subjectname
\keywords{} % Keywords for your thesis, this is not currently used anywhere in the template, print it elsewhere with \keywordnames
\university{\href{http://www.fh-aachen.de}{FH Aachen}} % Your university's name and URL, this is used in the title page and abstract, print it elsewhere with \univname
\department{\href{http://department.university.com}{Fachbereich 5}} % Your department's name and URL, this is used in the title page and abstract, print it elsewhere with \deptname
\group{\href{http://researchgroup.university.com}{Research Group Name}} % Your research group's name and URL, this is used in the title page, print it elsewhere with \groupname
\faculty{\href{https://www.fh-aachen.de/fachbereiche/elektrotechnik-und-informationstechnik/}{Fachbereich Name}} % Your faculty's name and URL, this is used in the title page and abstract, print it elsewhere with \facname

\AtBeginDocument{
\hypersetup{pdftitle=\ttitle} % Set the PDF's title to your title
\hypersetup{pdfauthor=\authorname} % Set the PDF's author to your name
\hypersetup{pdfkeywords=\keywordnames} % Set the PDF's keywords to your keywords
}

\begin{document}

\frontmatter % Use roman page numbering style (i, ii, iii, iv...) for the pre-content pages

\pagestyle{plain} % Default to the plain heading style until the thesis style is called for the body content

%----------------------------------------------------------------------------------------
%	TITLE PAGE
%----------------------------------------------------------------------------------------



\begin{titlepage}
\thispagestyle{empty}%
\setlength{\oddsidemargin}{0cm}%
\enlargethispage{\baselineskip}
\begin{textblock}{16}(188,8)%
    \includegraphics[width=1.5cm]{Figures/fh_logo_rechts.png}%
\end{textblock}%
\vspace*{-1.0cm}
\noindent
\LARGE\textbf{\univname}\\
%\Large\textbf{UNIVERSITY OF APPLIED SIENCES}\\
\Large\textbf{Fachbereich Elektrotechnik und Informationstechnik}\\
%\large \href{https://www.fh-aachen.de/menschen/hoever/}{\supname} 
\vspace{2cm}
\begin{center}
	\LARGE\textbf{Masterarbeit}\\
	\vspace{1.75cm}
	\LARGE \ttitle\\
	\large
	\vspace{2.5cm}
	Eingereicht von:\\
	{\authorname}\\
	Matrikelnummer: 3024062\\
	\vspace{1.5cm}
	Studiengang: Information Systems Engineering\\
	\vspace{1.5cm}
	\today\\
	\vspace{2cm}
	Betreuer: \href{https://www.fh-aachen.de/menschen/hoever/}{\supname}\\
	Korreferent: \href{https://www.fh-aachen.de/menschen/schuba/}{Prof. Dr. Marco \textsc{Schuba}}
\end{center}
\end{titlepage}

%----------------------------------------------------------------------------------------
%	DECLARATION PAGE
%----------------------------------------------------------------------------------------

\begin{declaration}
\vspace*{.04\textheight}
\noindent Ich versichere hiermit, dass ich die vorliegende Arbeit selbstständig verfasst und keine anderen als
die im Literaturverzeichnis angegebenen Quellen benutzt habe.\newline\newline
Stellen, die wörtlich oder sinngemäß aus veröffentlichten oder noch nicht veröffentlichten Quellen
entnommen sind, sind als solche kenntlich gemacht.\newline\newline
Die Zeichnungen oder Abbildungen in dieser Arbeit sind von mir selbst erstellt worden oder mit
einem entsprechenden Quellennachweis versehen.\newline\newline
Diese Arbeit ist in gleicher oder ähnlicher Form noch bei keiner anderen Prüfungsbehörde eingereicht
worden.\newline\newline\newline

\noindent Datum:\\
\rule[0.5em]{25em}{0.5pt} % This prints a line to write the date


\noindent Unterschrift:\\
\rule[0.5em]{25em}{0.5pt} % This prints a line for the signature
 
\end{declaration}

\cleardoublepage

%----------------------------------------------------------------------------------------
%	QUOTATION PAGE
%----------------------------------------------------------------------------------------

%\vspace*{0.2\textheight}

%\noindent\enquote{\itshape Thanks to my solid academic training, today I can write hundreds of words on virtually any topic without possessing a shred of information, which is how I got a good job in journalism.}\bigbreak

%\hfill Dave Barry

%----------------------------------------------------------------------------------------
%	ABSTRACT PAGE
%----------------------------------------------------------------------------------------

\begin{abstract}
\addchaptertocentry{\abstractname} % Add the abstract to the table of contents

Diese Zusammenfassung werde ich erst am Ende schreiben. Also nicht die Idee beschreiben, sondern eher zusammengefasst, was gemacht wurde und welche Resultate aus der Arbeit hervorgehen.

\end{abstract}

%----------------------------------------------------------------------------------------
%	ACKNOWLEDGEMENTS
%----------------------------------------------------------------------------------------

%\begin{acknowledgements}
%\addchaptertocentry{\acknowledgementname} % Add the acknowledgements to the table of contents
%The acknowledgments and the people to thank go here, don't forget to include your project advisor\ldots
%\end{acknowledgements}

%----------------------------------------------------------------------------------------
%	Table of content
%----------------------------------------------------------------------------------------

\setcounter{tocdepth}{2}
\tableofcontents % Prints the main table of contents


%----------------------------------------------------------------------------------------
%	ABBREVIATIONS
%----------------------------------------------------------------------------------------

\begin{abbreviations}{ll} % Include a list of abbreviations (a table of two columns)
\textbf{ECDSA} & \textbf{E}lliptic \textbf{C}urve \textbf{D}igital \textbf{S}ignature \textbf{A}lgorithm\\
\textbf{ASIC} & \textbf{A}pplication \textbf{S}pecific \textbf{I}ntegrated \textbf{C}ircuit\\
\textbf{BIP} & \textbf{B}itcoin  \textbf{I}mprovement \textbf{P}roposal\\
\textbf{GUI} & \textbf{G}raphical \textbf{U}ser \textbf{I}nterface\\
\textbf{RPC} & \textbf{R}emote  \textbf{P}rocedure \textbf{C}all\\
\textbf{EIP} & \textbf{E}thereum  \textbf{I}mprovement \textbf{P}roposal\\
\textbf{ABI} & \textbf{A}pplication  \textbf{B}inary \textbf{I}nterface\\
\textbf{JSON} & \textbf{J}avaScript \textbf{O}bject \textbf{No}tation\\


\end{abbreviations}

%----------------------------------------------------------------------------------------
%	PHYSICAL CONSTANTS/OTHER DEFINITIONS
%----------------------------------------------------------------------------------------

%\begin{constants}{lr@{${}={}$}l} % The list of physical constants is a three column table

% The \SI{}{} command is provided by the siunitx package, see its documentation for instructions on how to use it

%Speed of Light & $c_{0}$ & \SI{2.99792458e8}{\meter\per\second} (exact)\\
%Constant Name & $Symbol$ & $Constant Value$ with units\\

%\end{constants}

%----------------------------------------------------------------------------------------
%	SYMBOLS
%----------------------------------------------------------------------------------------

%\begin{symbols}{lll} % Include a list of Symbols (a three column table)

%$a$ & distance & \si{\meter} \\
%$P$ & power & \si{\watt} (\si{\joule\per\second}) \\
%Symbol & Name & Unit \\

%\addlinespace % Gap to separate the Roman symbols from the Greek

%$\omega$ & angular frequency & \si{\radian} \\

%\end{symbols}

%----------------------------------------------------------------------------------------
%	DEDICATION
%----------------------------------------------------------------------------------------

%\dedicatory{For/Dedicated to/To my\ldots} 

%----------------------------------------------------------------------------------------
%	THESIS CONTENT - CHAPTERS
%----------------------------------------------------------------------------------------

\mainmatter % Begin numeric (1,2,3...) page numbering

\pagestyle{thesis} % Return the page headers back to the "thesis" style

% Include the chapters of the thesis as separate files from the Chapters folder
% Uncomment the lines as you write the chapters

\chapter{Einleitung} % Main chapter title
\label{Chapter1}
%----------------------------------------------------------------------------------------
\newcommand{\keyword}[1]{\textbf{#1}}
\newcommand{\tabhead}[1]{\textbf{#1}}
\newcommand{\code}[1]{\texttt{#1}}
\newcommand{\file}[1]{\texttt{\bfseries#1}}
\newcommand{\option}[1]{\texttt{\itshape#1}}
%----------------------------------------------------------------------------------------
Die Erfindung der Kryptowährung Bitcoin und deren inhärente Blockchain-Technologie hat in den letzten Jahren einen regelrechten Hype ausgelöst. Begriffe wie Blockchain, Distributed Ledger und Smart Contract sind das bestimmende Thema schlechthin. 
Blockchain-Technologie verspricht, durch den Einsatz von Dezentralität, Unveränderlichkeit und Transparenz, die Möglichkeit zwischen anonymen, sich gegenseitig nicht trauenden, Parteien digitale Werte auszutauschen.

\section{Motivation}
Das Internet vernetzt weltweit mehrere Millionen Menschen. Es ermöglicht diesen anonym zu kommunizieren und miteinander zu interagieren. Aufgrund der Anonymität und des damit einhergehenden Betrugspotential, benötigen die meisten Anwendungsfälle eine sogenannte Trusted Third Party als Mittelsmann. Das fehlende Vertrauen zwischen den anonym agierenden Parteien wird durch diese dritte, nicht anonym auftretende, Partei kompensiert. Diese nicht anonyme Partei muss sich den länderspezifischen Regeln und Gesetzen unterwerfen und kann mit ihrem Service daher nur in einem gewissen, vorgegebenen Rahmen betreiben. Da das Betreiben eines solchen Services mit gewissen Kosten verbunden ist, werden diese meist auf die Nutzer abgewälzt. 

Online Casinos sind ein Beispiel für eine solche Trusted Third Party. Diese bieten eine Plattform, die Spieler vernetzt und vor gegenseitigem Betrug schützt. Außerdem verwalten Online Casinos das Geld ihrer Spieler. Die Spieler vertrauen den Casinos diese Aufgabe an, da es sich um registrierte Firmen handelt, die juristisch haftbar gemacht werden können. Ein weiterer Aspekt bei dem Vertrauen eine Rolle spielt sind die angebotenen Spiele selbst. Egal ob Black Jack, Poker oder Roulette, die meisten solcher Spiele erfordern die Generierung von Zufallszahlen. Beispielsweise wenn Spielkarten verteilt werden. Die Spieler haben dabei keine Möglichkeit nachzuprüfen, ob der Algorithmus, der ihnen die Karten zuteilt, auch wirklich fair ist. Die Spieler müssen dem Casino somit vertrauen, dass dieses sie nicht benachteiligt. 

%Der Endnutzer wird somit durch Transparenz und nicht durch staatliche Regulierungen geschützt. 
\section{Ziel}
Ziel dieser Masterarbeit ist es den Einsatz von Blockchain-Technologie an der beispielhaften Realisierung einer Glücksspielanwendung zu demonstrieren. Der Einsatz einer Blockchain soll dabei das Vertrauen, das der Endnutzer der Anwendung entgegenbringen muss, auf ein Minimum reduzieren. Falls möglich soll vollständig auf den Einsatz einer Trusted Third Party verzichtet werden.


%Das Ziel von Blockchains ist es durch transparente Systeme, die Interaktionen zwischen sich misstrauenden Parteien zu ermöglichen ohne, dass dabei das Vertrauen in eine Drittpartei erforderlich ist.

% Hier ein paar unwissenschaftliche Gründe warum 
%Als Spieler muss man sich erst registrieren und seine Bankdaten angeben. Durch diese Registrierung gibt der Spieler seine Anonymität auf. Außerdem hat der Staat eine zentrale Stelle an die er sich wenden kann um Steuereinnahmen durch die Besteuerung von Glücksspiel zu generieren. Ein weiterer Punkt der bei derzeitig etablierten Glücksspielplattformen verbesserungswürdig ist, sind Ein- und Auszahlungen. Einzahlungen und Auszahlungen  per Banküberweisungen dauern mindestens einen Tag. Dadurch muss der Glücksspieler eine lange Zeit warten bevor er den Service nutzen kann. Auch potentielle Gewinne kann er erst nach solch einer langen Zeitdauer weiterverwenden. Heutzutage gibt es zwar Dienste wie PayPal, die schnelle Zahlungen realisieren, allerdings kostet die Integration solch eines Zahlungsmittels recht hohe Gebühren für die Glücksspielplattform. Cryptowährungen bieten in dieser Hinsicht Abhilfe, da Transaktionen innerhalb wenigen Minuten bestätigt werden. Der Glücksspieler kann beispielsweise in einem Crypto-Casino spielen und sich anschließend von seinen Gewinnen eine Pizza bei Lieferando kaufen. :) (Hier nochmal drüber nachdenken ob man die Argumente nicht wissenschaftlicher verpacken könnte)

\section{Projektidee}
Die in dieser Arbeit betrachtete Glücksspielanwendung soll ein Spiel anbieten, bei dem Teilnehmer in einen Geldtopf einzahlen und auf ein zufälliges Event wetten. Jeder der Teilnehmer soll dabei die gleichen Gewinnchancen haben. Sobald alle Teilnehmer eingezahlt haben, wird einer der Teilnehmer zufällig ausgewählt und gewinnt den gesamten Geldtopf. Der Gewinner bekommt somit seinen eigenen Einsatz als auch den Einsatz aller Mitspieler ausgezahlt. Die restlichen Teilnehmer verlieren und gehen leer aus.\\\\

\noindent Die erstmalig in Bitcoin verwendete Blockchain Technologie ist für die Entwicklung einer solchen Anwendung bestens geeignet, da sie transparente, pseudonyme Zahlungen ermöglicht. Außerdem lässt sich der für die Gewinnerauswahl benötigte Zufall durch ein in der Zukunft liegenden Zustand der Blockchain abbilden. 
Der Zufallsfaktor kommt somit direkt von der Blockchain und ist für alle Teilnehmer nachvollziehbar.

%Was genau das eigentliche Spiel für den Nutzer der Glücksspielanwendung ist, ist für diese Masterarbeit zweitrangig. Ziel dieser Masterarbeit ist es zu erforschen, welche Möglichkeiten die erstmals in Bitcoin verwendete Blockchain Technologie bietet und in wie weit durch sie das Vertrauen des Endnutzers in die Anwendung reduziert werden kann.


\section{Anforderungen}\label{anforderungen}
Die zu realisierende Glücksspielanwendung muss den folgenden Anforderungen gerecht werden.
\subsubsection{1) Transparente Einzahlungen}
Die Einzahlung jedes Endnutzers ist für jeden anderen Endnutzer nachprüfbar.
\subsubsection{2) Gewinnerauswahl durch Zufallsfaktor}
Die Auswahl des Gewinners ist von einem zufälligen Faktor abhängig, auf den weder die Anwendung noch die Endnutzer einen Einfluss haben.
\subsubsection{3) Nachprüfbarkeit des Zufallsfaktor}
Jeder Endnutzer kann die Echtheit des zufälligen Faktors eigenständig nachprüfen.
\subsubsection{4) Transparente Auszahlungen}
Die Auszahlung an den Gewinner muss transparent und somit für jeden Endnutzer nachprüfbar sein.
\subsubsection{5) Fairheit des Spiels}
Jeder Endnutzer besitzt die gleiche Gewinnwahrscheinlichkeit und niemand wird benachteiligt.

\section{Vorhandenes}

\subsection{Cyberdice Protokoll}

Einen ersten Ansatz wie man im Internet Glücksspiel ohne eine vertrauenswürdige Drittpartei betreiben kann, liefert \cite{cyberdice_paper}. Es stellt ein Kommunikationsprotokoll vor, das mit Hilfe kryptographischer Methoden sicherstellt, dass weder die Teilnehmer noch Außenstehende betrügen können. Das zum Glücksspiel verwendete Protokoll funktioniert aber nur unter der Annahme, dass es eine zentrale Institution (Bank) gibt, bei der die Teilnehmer Geld einzahlen und im Falle eines Gewinns gegen die Vorlage eines Beweises Geld ausgezahlt bekommen. Durch die Erfindung dezentraler Kryptowährungen, die auf einer für jeden einsehbaren Blockchain basieren, fällt diese vorher noch benötigte zentrale Institution weg.

%Durch die Erfindung dezentraler, Ländergrenzen überschreitender Krypto-Währungssysteme ist es möglich geworden den Einfluss des Staates zu umgehen und den Endnutzer durch die Transparenz der Glücksspielanwendung zu schützen.

\subsection{Glücksspielseiten}
Es gibt bereits Services die dezentrales, transparentes Glücksspiel mit Hilfe von Kryptowährungen umsetzen.

Die Internetseite \cite{crypto_games} bietet diverse Spiele an, bei denen die Nutzer die Möglichkeit haben mit Kryptowährungen zu bezahlen.

Die Internetseite \cite{vdice} bietet ein Würfelspiel an, das durch einen Smart Contract auf der Ethereum Plattform umgesetzt ist.

Eine genaue Beschreibung der verwendeten Verfahren zur Gewinnerauswahl befindet sich im Anhang dieser Ausarbeitung.

\section{Aufbau dieser Arbeit}
In dieser Arbeit wird zunächst Bitcoin als Beispiel einer auf Blockchain-Technologie basierenden Kryptowährung betrachtet. Anschließend wird der Einsatz von Smart Contracts mit Ethereum präsentiert. In beiden Fällen werden zunächst die relevanten Grundlagen geklärt und ein Konzept vorgestellt. Anschließend wird dieses als Glücksspielanwendung mithilfe der jeweiligen Technologie umgesetzt. Die resultierenden Glücksspielanwendungen werden letztendlich unter Zuhilfenahme der aufgestellten Anforderungen evaluiert. Nach diesem Hauptteil wird abschließend weitere Blockchain-Technologien und deren Anwendbarkeit für die Projektidee betrachtet. 

\chapter{Erster Ansatz: Bitcoin}
\label{btc}

\section{Grundlagen}
\label{btc_grundlagen}
Bei Bitcoin handelt es sich um die erste digitale, dezentral organisierte Währung. Die Idee digitaler Währungen existiert bereits seit der Erfindung des Internets. Allerdings scheiterten diese in der Vergangenheit daran, dass sie auf einen zentralen Punkt der Kontrolle angewiesen waren und somit einen Single Point of Failure beinhalteten. Die am 03. Januar 2009 gestartete digitale Währung ''Bitcoin'' schaffte es erstmalig gänzlich auf die Verwendung einer zentralen Instanz zu verzichten und somit ein verteiltes, dezentrales und sicheres digitales Zahlungssystem zu realisieren. Bitcoin wurde in dem von dem Pseudonym ''Satoshi Nakamoto'' veröffentlichen Paper ''Bitcoin: A Peer-to-Peer Electronic Cash System'' das erste Mal beschrieben. \if Der Source Code der Bitcoin Software ist öffentlich verfügbar und hat dadurch seit 2009 eine regelrechte Innovationsexplosion im Bereich der digitalen Währungen ausgelöst.\fi Bitcoin funktioniert durch das Zusammenspiel mehrerer Komponenten und besteht aus:

\begin{itemize}
\item Einem dezentralen \textit{Peer-to-Peer Netzwerk}, das mit Hilfe des Bitcoin-Protokolls kommuniziert.
\item Der \textit{Blockchain}, die eine öffentlichen Transaktionsdatenbank darstellt, die alle validen Transaktionen seit dem Start des Netzwerkes aufzeichnet.
\item Einer Menge an \textit{Konsensregeln} mit Hilfe derer Netzwerkteilnehmer eigenständig Transaktionen auf ihre Gültigkeit prüfen können.
\item Einem \textit{Proof-of-Work Algorithmus}, der es erlaubt, sich in dem globalen dezentralen Netzwerk auf den Zustand der Transaktionsdatenbank zu einigen. Das kontinuierliche Ausführen des Proof-of-Work Algorithmus wird \textit{Mining} genannt.
\end{itemize}

\subsection{Peer-to-Peer Netzwerk}
Peer-to-Peer-Netzwerke sind Netzwerke, die auf direkten Verbindungen zwischen Rechnern beruhen, ohne dass dabei einer der Rechner eine Sonderstellung einnimmt oder ein Server die Kommunikation vermittelt. In einem reinen Peer-to-Peer-Netz sind alle Computer gleichberechtigt und können sowohl Dienste in Anspruch nehmen, als auch zur Verfügung stellen. Das Peer-to-Peer-Modell ist somit grundlegend verschieden von dem im Internet am häufigsten verwendeten Client-Server-Modell. Da jeder Knoten des Netzwerks gleichzeitig Client und Server ist, gibt es keine zentrale Instanz, die einen sogenannten ''Single Point of Failure'' darstellt. 

\begin{figure}[H]
\centering
\includegraphics[width=1\linewidth]{Figures/p2p_networks}
\decoRule
\caption{Client-Server | Peer-to-Peer \cite{wikipedia_p2p}}
\label{fig:p2p_networks}
\end{figure}

Peer-to-Peer Netzwerke sind selbstorganisierend. Das Hinzufügen neuer Teilnehmer und das Entfernen bestehender Netzwerkknoten findet ohne eine zentrale Verwaltung statt und behindert die Funktionsweise des Netzwerks nicht. Jeder Knoten des Netzwerks verwaltet eigenständig seine direkten Nachbarknoten. Die Art und Weise wie die Teilnehmer des Netzwerks miteinander kommunizieren ist durch das Netzwerkprotokoll vorgegeben. Um am Netzwerk teilzunehmen braucht man nur eine Software, die das Netzwerkprotokoll implementiert und einen Internetanschluss. Im Falle der Kryptowährungen nennt man die Software ''Wallet'' (englisch für Brieftasche), da man mit ihr Zahlungen initiieren und empfangen kann. Möchte ein Teilnehmer Bitcoins an einen anderen Teilnehmer senden, erstellt er dazu eine Nachricht die solch eine Transaktion beinhaltet und schickt sie an seine direkten Nachbarn des Peer-to-Peer Netzwerks. Die Nachbarn prüfen die Gültigkeit der Transaktion leiten diese gegebenenfalls an ihre Nachbarn weiter. Auf diese Art und Weise verteilt sich eine Transaktion im gesamten Netzwerk.


\subsection{Blockchain}
Eine Blockchain ist eine global verteilte Transaktionsdatenbank. Jeder Teilnehmer des Peer-to-Peer Netzwerkes speichert lokal eine Kopie dieser Datenbank. Dies erlaubt es ihm jegliche Datenbankeinträge zu lesen. Im Gegensatz zum lesenden Zugriff ist der schreibende Zugriff auf die Datenbank nur unter sehr strikten Regeln möglich. Über diese Regeln sind sich alle Teilnehmer des Netzwerkes einig. Daher werden diese Regeln Konsensregeln genannt. Möchte ein Teilnehmer eine Transaktion in die Datenbank schreiben, muss er sicherstellen, dass sie den Konsensregeln entspricht. Falls die Transaktion eine Konsensregel bricht wird sie vom Netzwerk verworfen und es ist ausgeschlossen, dass Sie in die Blockchain aufgenommen wird. Eine Transaktion beschreibt den Übergang von einem alten Systemzustand in einen neuen Systemzustand.
Im Fall von Bitcoin handelt es sich bei dem Systemzustand um ein digitales Kontenbuch. Die Konten sind bei Bitcoin sogenannte Adressen und repräsentieren den öffentlichen Schlüssel eines ECDSA Schlüsselpaars. Um die einer Adresse zugeschriebenen Bitcoins zu überweisen, muss der Besitzer, mit Hilfe des privaten Schlüssels, eine digitale Signatur erstellen. Diese Signatur garantiert, dass die Überweisung vom Besitzer der Bitcoins autorisiert ist.

\begin{figure}[H]
\centering
\includegraphics[width=1\linewidth]{Figures/BTC_statetransition_ETH_white_paper}
\decoRule
\caption{Bitcoin Zustandsveränderung durch Transaktion}
\label{fig:BTC_statetransition_ETH_white_paper}
\end{figure}

Die Transaktion aus Abbildung \ref{fig:BTC_statetransition_ETH_white_paper} überweist 1 Bitcoin von der Adresse \code{7b53ab84} und 2 Bitcoin von Adresse \code{3ce6f712} auf die Adresse \code{bb75a980} und überführt somit das Kontobuch in einen neuen Zustand. Möchten mehrere Teilnehmer den Systemzustand durch Transaktionen gleichzeitig anpassen, spielt die Reihenfolge, in der die Transaktionen ausgeführt werden, eine wichtige Rolle. Aus diesem Grund werden Transaktionen in sogenannten Blöcken, in einer festen Reihenfolge, aggregiert. Somit werden nicht einzelne Transaktionen, sondern ganze Blöcke von Transaktionen in die Datenbank geschrieben. Genau wie bei den Transaktionen gibt es auch für Blöcke gewisse Konsensregeln. Sobald ein Block allen Konsensregeln entspricht, ist er bereit in die Datenbank aufgenommen zu werden.
Da sich alle Netzwerkteilnehmer über die Gültigkeit des Blockes einig sind, wird somit die globale Blockchain Datenbank angepasst. Genau wie bei den Transaktionen ist auch die Reihenfolge der Blöcke wichtig. Daher beinhaltet jeder Block den Hash-Wert seines Vorgängers(siehe \ref{fig:blockchain_ETH_white_paper}).

\begin{figure}[H]
\centering
\includegraphics[width=1\linewidth]{Figures/blockchain_ETH_white_paper}
\decoRule
\caption[Kette von Blöcken]{Verkettung von Blöcken.}
\label{fig:blockchain_ETH_white_paper}
\end{figure}

Durch die so erzielte Verkettung der Blöcke wird die Reihenfolge eindeutig festgelegt und es entsteht die sogenannte Blockchain. Am Anfang der Blockchain befindet sich der sogenannte Genesis Block. Auf diesen bauen alle weiteren Blöcke auf. Nachträgliche Änderungen an einem bereits eingefügtem Block sind nicht möglich, da sich dadurch der Blockhash des Blocks verändert und somit an dieser Stelle die Kette ''zerbricht''.

\subsection{Konsensregeln}

Konsenzregeln sind Regeln über die sich alle Teilnehmer des Peer-to-Peer-Netzwerks einig sind. Sie stellen sicher, dass die grundlegenden Eigenschaften der Kryptowährung eingehalten werden. 
\if
Protokollregeln legen die Syntax der ausgetauschten Nachrichten fest. Konsensregeln legen dagegen die Semantik der ausgetauschten Nachrichten fest. 
Konsensregeln sind den Protokollregeln übergeordnet. Dies bedeutet, dass eine laut Protokollregeln korrekt aufgebaute Transaktionsnachricht, dennoch eine ungültige Transaktion enthalten kann, die den Konsensregeln widerspricht.
\fi
Bei Bitcoin gibt es eine ganze Reihe an Konsensregeln. Die wichtigsten sind:
\begin{itemize}
\item Transaktionen dürfen kein Geld aus dem nichts schöpfen, sondern nur bereits existierende Beträge von einer Adresse auf eine andere Adresse überweisen. Die Blockreward-Konsensregel bildet hierzu die einzige Ausnahme. Sie legt fest wie neue Währungseinheiten erschaffen werden.
\item Blockreward: Ein neuer Block muss genau eine Transaktion enthalten, die neue Kryptowährungseinheiten aus dem nichts erschafft. Sowohl die Höhe des Betrags als auch die Anpassung des Betrags über die Zeit ist in den Konsensregeln des Protokolls hart verankert. Bei Bitcoin startete der Blockreward bei 50 Bitcoin und halbiert sich seitdem alle 4 Jahre. Dies stellt sicher, dass es eine feste Obergrenze an Währungseinheiten gibt.
\item Transaktionen, die Geld von Adresse A nach Adresse B überweisen, müssen durch eine Signatur beweisen, dass sie von dem rechtmäßigen Besitzer getätigt werden.
\item Blockgröße: Diese Konsensregel legt die maximale Größe eines Blocks in Megabyte fest. Sie beeinflusst wie viele Transaktionen in einem Block gebündelt werden können. Dies ist wichtig, da sie zusammen mit der Blockzeit-Konsensregel das Wachstum der Blockchain-Datenbank steuert.
\item Blockzeit: Diese legt fest in welchem durchschnittlich Zeitabstand es erlaubt ist einen neuen Block in die Blockhain einzufügen. Bei Bitcoin ist diese Zeit auf 10 Minuten festgelegt.
\item Längste Blockchain-Kette: Teilnehmer des Peer-to-Peer Netzwerks folgen immer der Kette, die am meisten Proof-of-Work beinhaltet und betrachten ''kürzere'' Ketten als ungültig.
\end{itemize}
Die Konsensregeln ermöglichen es, dass jeder Teilnehmer eigenständig lokal seine Version der Blockchain Datenbank verwalten kann, ohne dabei einem anderen Teilnehmer vertrauen zu müssen. Die Konsensregeln stellen somit sicher, dass alle Teilnehmer die gleiche Blockchain Datenbank lokal aufbauen und sich dadurch auf den Systemzustand einigen.

\if Dadurch herrscht Einigkeit darüber welche Transaktionen bereits in die Blockchain aufgenommen wurden und somit als bestätigt gelten. Alle noch nicht in die Blockchain aufgenommen Transaktionen gelten als unbestätigt. \fi

\subsection{Proof-of-Work und Mining}
Das Einfügen eines neuen Blocks in die Blockchain Datenbank ist nur unter sehr strikten Regeln möglich. Bei einer dezentralen Währung wie Bitcoin gibt es keine zentrale Instanz, die solch eine Anpassung des Systemzustands koordiniert, beziehungsweise autorisiert. Um dieses Problem zu lösen verwendet Bitcoin einen Proof-of-Work Algorithmus. Ziel des Algorithmus ist es durch das kontinuierliche\footnote{Das \code{nonce}-Feld des neusten Blocks wird nach jeder Berechnung des Hashwerts erhöht, damit ein neuer Hashwert entsteht.} Hashen des neusten Blocks einen Hashwert zu finden der unterhalb eines dynamisch angepassten Schwierigkeit-Zielwerts liegt. Der Schwierigkeit-Zielwert wird durch die Blockzeit Konsensregel so angepasst, dass im gesamten Netzwerk im Durchschnitt alle 10 Minuten ein neuer Bitcoin Block gefunden wird. Findet ein Teilnehmer den Blockhash eines den Konsensregeln entsprechenden Blocks, leitet er diesen Block an das Peer-to-Peer Netzwerk weiter und erhält im Gegenzug den Blockreward. Teilnehmer, die unbestätigte Transaktionen empfangen, diese in Blöcke zusammenfassen und unter Zuhilfenahme des Proof-of-Work Algorithmus in die Blockchain einfügen, werden ''Miner'' genannt. Dieser Name stammt daher, dass bei diesem rechenintensiven Prozess gleichzeitig neue Bitcoins erschaffen werden.

Beim Mining werden die in Tabelle \ref{tab:btc_block_header} aufgeführten Felder des Blockheaders als Eingabe für die kryptographische Hashfunktion SHA256 verwendet\footnote{Um genau zu sein wird der Blockheader bei Bitcoin zweimal mit der SHA 256 Hashfunktion gehasht. Blockhash = SHA256(SHA256(Blockheader))}.
\begin{table}[H]
\centering
\caption{Bitcoin Blockheader}
\label{tab:btc_block_header}
\begin{tabular}{|l|p{4,3cm}|p{4,3cm}|l|}
\hline
\textbf{Feld}  & \textbf{Verwendungszweck}                         & \textbf{Aktualisiert falls...}                                          & \textbf{Bytes} \\ \hline
version        & Block Versionsnumber                             & man die Software aktualisiert und diese eine neue Version spezifiziert. & 4              \\ \hline
hashPrevBlock  & 256-bit Hash des vorherigen Blockheaders          & ein neuer gültiger Block empfangen wird.                                & 32             \\ \hline
hashMerkleRoot & 256-bit Hash aller Transaktionen des Blocks       & eine neue Transaktion akzeptiert wird.                                  & 32             \\ \hline
time           & Aktueller Zeitstemple in Sekunden seit 1970-01-01 & Alle paar Sekunden...                                                   & 4              \\ \hline
bits           & Aktuelles Schwierigkeitsziel                      & wenn das Schwierigkeitsziel angepasst wird.                             & 4              \\ \hline
nonce          & 32-bit Nummer (von 0 aus erhöht)                  & der Hash ausprobiert wurde. (nonce+1)                                   & 4              \\ \hline
\end{tabular}
\end{table}
Die Transaktionen des Blocks gehen nicht direkt, nur durch den sogenannten Merkel Tree Hash\footnote{Der genaue Aufbau des Merkel Trees und dessen Vorteile sind in \cite{bitcoin_white_paper} genauer beschrieben.} in den Blockhash mit ein. Das \code{nonce}-Feld des Blockheaders wird nach jedem Hashversuch um den Wert 1 erhöht. Dadurch wird die Ausgabe der Hashfunktion kontinuierlich verändert.

Die Sicherheit des Bitcoin Netzwerkes und die Unmanipulierbarkeit der Blockchain ergeben sich aus der im Protokoll verankerten Spieltheorie. Die Spieltheorie macht es für einen Miner profitabler sich an die Spielregeln des Protokolls zu halten als zu versuchen das Netzwerk zu betrügen. Versucht ein Miner einen Block zu produzieren, der den Konsensregeln widerspricht, wird dieser vom Netzwerk verworfen. Somit hält kein anderer Netzwerkteilnehmer den vom Miner an sich selbst ausgeschütteten Blockreward für gültig. Da der Miner aber Ausgaben in Form von Hardwareabnutzung und Stromkosten zu bezahlen hat, schafft dies einen Anreiz sich den Konsensregeln zu unterwerfen. Solange 51 Prozent der Miner sich den Konsensregeln unterwerfen, formen diese auf Dauer die längste Blockchain. Die beim Mining anfallenden Stromkosten machen es außerdem sehr Teuer die Geschichte der Blockchain neu zu schreiben.
Das Mining ist ein sich selbst regulierendes System bei dem die Anzahl Miner\footnote{Genauer gesagt nicht die Anzahl Miner, sondern die von den Minern erbrachte Rechenleistung.} sich mit dem Wert der Kryptowährung anpasst. Steigt der Preis pro Bitcoin, kommen neue Miner zum Netzwerk hinzu und machen das Netzwerk sicherer. Ein sinkender Preis hat zur Folge, dass der Blockreward nicht mehr für die Bezahlung der Strom und Hardwarekosten ausreicht. Dies hat zur Folge, dass die Anzahl an Miner abnimmt. Somit sinkt auch die Sicherheit des Netzwerks. 
\section{Konzept}

Die folgenden Schritte beschreiben den Finanzfluss zwischen den Teilnehmern und der Anwendung sowie die Gewinnerauswahl durch den in der Zukunft liegenden Blockchain-Status. Der Ablauf ist allgemein gehalten und kann nicht nur mit Bitcoin, sondern auch mit anderen Kryptowährungen, die auf einer Proof-of-Work Blockchain basieren, umgesetzt werden. Betrachtet wird ein Spiel mit N Teilnehmern, bei dem jeder Teilnehmer einen Einsatz von X Währungseinheiten zur Teilname zahlen muss. 

\begin{enumerate}
\item Im ersten Schritt eröffnet die Anwendung ein neues Spiel in dem es N freie Plätze und einen leeren Geldtopf gibt.
\item Sobald ein Spieler am Spiel teilnehmen möchte, generiert die Anwendung eine neue Empfangsadresse und zeigt diese dem Spieler an. 
\item Der Spieler verwendet die Wallet Software seiner Wahl um eine Transaktion zu erstellen, die den Einsatz an die angezeigte Empfangsadresse überweist. Die Wallet Software signiert die Transaktion und leitet sie über die mit ihr verbundenen Nachbarn an das Peer-to-Peer Netzwerk weiter.
\item Ein Miner empfängt die Transaktion und nimmt sie in den nächsten Block auf.
\item Der Miner findet den zu seinem Block passenden Proof-of-Work-Hash und schickt den Block an das Netzwerk.
\item Die Applikation empfängt den Block und merkt, dass im Block eine Transaktion auf die in Schritt 2 generierte Empfangsadresse enthalten ist. Die Applikation prüft die Höhe des Transaktionsbetrag und leitet anschließend die vom Spieler kontrollierte Auszahlungsadresse aus der Transaktion ab. 
\item  Die restlichen N-1 Teilnehmer überweisen ebenfalls den geforderten Betrag auf die ihnen angezeigte Empfangsadresse.
\item  Sobald die letzte Transaktion in einen validen Block aufgenommen wurde, zählt die Reihenfolge in der die Transaktionen in der Blockchain stehen. Die Reihenfolge steht somit fest und kann nicht mehr nachträglich verändert werden. Der Geldtopf ist nun mit einem Betrag von N*X Krytowährungseinheiten gefüllt und wird geschlossen. Der Block nach dem Block, in dem die letzte Einzahlungstransaktion eingegangen ist, wird zur Gewinnerauswahl genutzt. Die Anwendung merkt sich die Blocknummer dieses Blocks.
%\item Die Anwendung signiert nun die Reihenfolge der Teilnehmer und deren Auszahlungsadressen und schreibt die resultierende Signatur mit einer Transaktion in die Blockchain.
\item Die Anwendung und die Teilnehmer warten darauf, dass der nächste Block von einem Miner gefunden wird. Alle Miner des Peer-to-Peer Netzwerks versuchen schnellstmöglich einen passenden Blockhash zu finden um den Blockreward zu erhalten. Ein Miner gewinnt dieses Rennen und teilt dem Netzwerk den neu gefundenen Block mit.
\item Die Anwendung empfängt den nächsten Block und ermittelt durch diesen den Gewinner. Die Berechnung erfolgt indem die Anwendung den Blockhash des Blocks vom Hexadezimalsystem ins Dezimalsystem konvertiert. Der dabei resultierende sehr hoher Wert B bildet die Grundlage für die Gewinnerauswahl. Durch die Berechnung von B modulo N resultiert eine Zahl G zwischen 0 und N-1 die den Gewinner festlegt. Der Spieler der die G+1te Einzahltransaktion gesendet hat, gewinnt den Geldtopf.
\item Die Anwendung erstellt eine Transaktion, die alle N*X Krytowährungseinheiten des Geldtopfs an die Auszahlungsadresse des Gewinners überweist, und sendet diese an das Netzwerk.
\item Die Wallet Software des Gewinners, empfängt die Transaktion und informiert den Teilnehmer, dass er den gesamten Betrag des Topfes erhalten hat.
\end{enumerate}

%%%%%%%%%%%%%%%%%%%%
% Bitcoin Beispiel %
%%%%%%%%%%%%%%%%%%%%
\vspace{1cm}
Im folgenden Beispiel wird einen Topf mit 5 Teilnehmern, die Kryptowährung Bitcoin und einen Einzahlungsbetrag von 0,1 Bitcoin betrachtet. Dieses Beispiel verdeutlicht sowohl die Interaktion der verschiedenen Teilnehmer des Peer-to-Peer Netzwerks, als auch die Veränderung des Status der Blockchain.

\vspace{1cm}
\begin{minipage}{0.55\textwidth}
\includegraphics[width=\textwidth]{Figures/konzept_btc/konzept1}
\centering
\decoRule
\captionof{figure}{\label{fig:konzept1}Schritt 1}
\end{minipage}
\begin{minipage}{0.45\textwidth}
Diese Abbildung zeigt das Peer-To-Peer Netzwerk. Die 5 potentiellen Teilnehmer sind durch Notebooks und Smartphones dargestellt. Außerdem sind 2 Miner und die Glücksspielanwendung teil des Peer-To-Peer Netzwerks. Der aktuelle Status der Blockchain, die jeder Teilnehmer des Netzwerks lokal speichert ist unterhalb des Netzwerkes dargestellt.
\end{minipage}

\vspace{1cm}
\begin{minipage}{0.55\textwidth}
\includegraphics[width=\textwidth]{Figures/konzept_btc/konzept2}
\centering
\decoRule
\captionof{figure}{\label{fig:konzept2}Schritt 2}
\end{minipage}
\begin{minipage}{0.45\textwidth}
Die Bitcoin Client Software der Glücksspielanwendung generiert eine neue Bitcoinadresse und speichert den dazugehörigen privaten Schlüssel in der Wallet. Sobald Bitcoins auf dieser Adresse empfangen werden, können sie nur durch den Besitz des privaten Schlüssels weiter transferiert werden.
Die Anwendung zeigt dem Benutzer eine frisch generierte Empfangsadresse über die Benutzeroberfläche an. Der Zustand der Blockchain verändert sich nicht.
\end{minipage}

\vspace{1cm}
\begin{minipage}{0.55\textwidth}
\includegraphics[width=\textwidth]{Figures/konzept_btc/konzept3}
\centering
\decoRule
\captionof{figure}{\label{fig:konzept3}Schritt 3}
\label{fig:konzept3}
\end{minipage}
\begin{minipage}{0.45\textwidth}
Nun zahlt der Spieler mit Hilfe seiner Bitcoin Wallet Software in den Geldtopf ein. Dazu erstellt er eine Transaktion, die Bitcoin von seiner Adresse auf die generierte Adresse der Glücksspielanwendung transferiert. Durch die Signierung mit seinem privaten Schlüssel autorisiert er die Überweisung. Anschließend schickt er die Transaktion seinen Nachbarn.
\end{minipage}

\vspace{1cm}
\begin{minipage}{0.55\textwidth}
\includegraphics[width=\textwidth]{Figures/konzept_btc/konzept4}
\centering
\decoRule
\captionof{figure}{\label{fig:konzept4}Schritt 4}
\label{fig:konzept4}
\end{minipage}
\begin{minipage}{0.45\textwidth}
Sobald die Transaktion TXN 1 einen Miner erreicht, prüft dieser ob die Transaktion in Einklang mit den Konsensregeln ist. In diesem Beispiel existiert in Block 9 eine Transaktion von einem Bitcoin auf die Adresse des Teilnehmers. Unter der Annahme, dass dieser Bitcoin nicht in Block 10 weiter überwiesen wurde, befindet sich auf der Adresse des Teilnehmers somit ein Bitcoin. Außerdem prüft der Miner ob die Signatur der Transaktion gültig ist. Da die Transaktion valide ist, fügt er sie dem aktuell zu generierenden Block Nummer 11 hinzu. 
\end{minipage}


\vspace{1cm}
\begin{minipage}{0.55\textwidth}
\includegraphics[width=\textwidth]{Figures/konzept_btc/konzept5}
\centering
\decoRule
\captionof{figure}{\label{fig:konzept5}Schritt 5}
\label{fig:konzept5}
\end{minipage}
\begin{minipage}{0.45\textwidth}
Der Miner berechnet nun mithilfe der SHA256 Hashfunktion den Hash des Blocks. Falls der Blockhash-Wert den durch die Konsensregeln dynamischen angepassten Schwierigkeits-Wert unterschreitet, gilt der Block als valide. Überschreitet der Blockhash den Wert, erhöht der Miner den Nonce-Wert des Blocks und berechnet den Blockhash erneut. Diesen Prozess wiederholt er solange bis er entweder einen gültigen Blockhash findet oder einen gültigen Block Nummer 11 von einem anderen Netzwerkteilnehmer empfängt.
In diesem Beispiel findet der Miner einen gültigen Blockhash, leitet den Block ans Netzwerk weiter und wird dadurch mit neu erschaffenen Bitcoin für seinen Rechenaufwand belohnt.
\end{minipage}

\vspace{1cm}
\begin{minipage}{0.55\textwidth}
\includegraphics[width=\textwidth]{Figures/konzept_btc/konzept6}
\centering
\decoRule
\captionof{figure}{\label{fig:konzept6}Schritt 6}
\label{fig:konzept6}
\end{minipage}
\begin{minipage}{0.45\textwidth}
Die Glücksspielanwendung empfängt den Block Nummer 11 und überprüft ob er im Einklang mit den Konsensregeln ist. Dies ist der Fall. Somit wird die lokale Blockchain Datenbank um einen Block erweitert. Die Glücksspielanwendung hat somit den Einsatz des ersten Spielers erhalten. Aus der Einzahlungstransaktion des Spielers leitet die Anwendung die Auszahlungsadresse \textbf{1Axjc9} des Spielers ab. 
\end{minipage}

\vspace{1cm}
\begin{minipage}{0.55\textwidth}
\includegraphics[width=\textwidth]{Figures/konzept_btc/konzept7}
\centering
\decoRule
\captionof{figure}{\label{fig:konzept7}Schritt 7}
\label{fig:konzept7}
\end{minipage}
\begin{minipage}{0.45\textwidth}
Die restlichen Spieler senden ihre signierten 0,1 Bitcoin Transaktionen ins Peer-to-Peer Netzwerk. Diese sind in Abbildung 7 durch die Transaktionen TXN 2 bis 5 dargestellt.
\end{minipage}

\vspace{1cm}
\begin{minipage}{0.55\textwidth}
\includegraphics[width=\textwidth]{Figures/konzept_btc/konzept8}
\centering
\decoRule
\captionof{figure}{\label{fig:konzept8}Schritt 8}
\label{fig:konzept8}
\end{minipage}
\begin{minipage}{0.45\textwidth}
Beliebige Miner fügen die Transaktionen in ihre Blöcke ein. Sobald die Glücksspielanwendung die Blöcke empfängt, prüft sie diese gegen die Konsensregeln und fügt sie in die lokale Blockchain ein.
Die Applikation merkt nun, dass alle Spieler bezahlt haben und schließt den Geldtopf. Dabei merkt sie sich die Nummer des Blocks in der die letzte Einzahlungstransaktion vorhanden ist. Der darauffolgende Block mit Nummer 14 wird für die Ziehung des Gewinners verwendet.
\end{minipage}

\vspace{1cm}
\begin{minipage}{0.55\textwidth}
\includegraphics[width=\textwidth]{Figures/konzept_btc/konzept9}
\centering
\decoRule
\captionof{figure}{\label{fig:konzept9}Schritt 9}
\label{fig:konzept9}
\end{minipage}
\begin{minipage}{0.45\textwidth}
Alle Miner des Netzwerkes versuchen nun gleichzeitig so schnell wie möglich den nächsten Block zu finden. Da sie dazu eine kryptographische Hashfunktion benutzen bei der die Ausgabe ein unkontrollierbarer zufälliger Wert ist, hat keiner der Miner einen direkten Einfluss auf den resultierenden Blockhash. In diesem Beispiel findet Miner 2 einen gültigen Blockhash vor Miner 1. Miner 2 leitet seinen gültigen Block Nummer 14 so schnell wie möglich an das Netzwerk weiter und erhält den Blockreward als Belohnung. Miner 1 geht leer aus.
\end{minipage}

\vspace{1cm}
\begin{minipage}{0.55\textwidth}
\includegraphics[width=\textwidth]{Figures/konzept_btc/konzept10}
\centering
\decoRule
\captionof{figure}{\label{fig:konzept10}Schritt 10}
\label{fig:konzept10}
\end{minipage}
\begin{minipage}{0.45\textwidth}
Die Anwendung empfängt Block 14 und prüft ihn gegen die Konsensregeln. Der Block ist valide. Daher verwendet die Anwendung den im Block enthaltenen Blockhash um den Gewinner des Geldtopfes zu ermitteln. Statt des gesamten Blockhashs verwendet die Anwendung nur die letzte Ziffer des Blockhashs zur Gewinnerauswahl. Dies hat den Vorteil, dass die Teilnehmer die Korrektheit der Gewinnerauswahl leichter eigenständig nachprüfen können.
Da die letzte numerische Stelle des Blockhashs 10 verschiedene Werte annehmen kann, ordnet die Anwendung jedem der 5 Teilnehmer 2 Gewinnzahlen zu.
\end{minipage}

\vspace{1cm}
Dies erreicht die Anwendung indem sie die letzte Blockhash-Ziffer modulo 5 nimmt. Dadurch ergibt sich die Verteilung der Gewinnzahlen folgendermaßen:
\begin{itemize}
\item Spieler 1 mit Adresse 1xw9vA gewinnt bei 0 und 5,
\item Spieler 2 mit Adresse 1xw9vD gewinnt bei 1 und 6,
\item Spieler 3 mit Adresse 1xw9vB gewinnt bei 2 und 7,
\item Spieler 4 mit Adresse 1xw9vC gewinnt bei 3 und 8,
\item Spieler 5 mit Adresse 1xw9vE gewinnt bei 4 und 9.
\end{itemize}
Jeder Teilnehmer besitzt nun eine Gewinnwahrscheinlichkeit von 1/5. Block Nummer 14 hat den Blockhash \textbf{00000001A5}. Die zur Gewinnerauswahl benutze Ziffer ist somit die 5. Da 5 modulo 5 den Wert 0 ergibt, gewinnt Spieler 1 mit der Adresse \textbf{1xw9vA}.

\vspace{1cm}
\begin{minipage}{0.55\textwidth}
\includegraphics[width=\textwidth]{Figures/konzept_btc/konzept11}
\centering
\decoRule
\captionof{figure}{\label{fig:konzept11}Schritt 11}
\label{fig:konzept11}
\end{minipage}
\begin{minipage}{0.45\textwidth}
Die Anwendung erstellt nun eine Transaktion die alle Spieleinsätze an die Auszahlungsadresse \textbf{1Axjc9} von Spieler 1 überweist. Um die Transaktion zu signieren verwendet die Anwendung die, zu den 5 Einzahlungsadressen passenden, privaten Schüssel. Anschließend leitet die Anwendung die Transaktion an das Peer-to-Peer Netzwerk weiter.
\end{minipage}

\vspace{1cm}
\begin{minipage}{0.55\textwidth}
\includegraphics[width=\textwidth]{Figures/konzept_btc/konzept12}
\centering
\decoRule
\captionof{figure}{\label{fig:konzept12}Schritt 12}
\label{fig:konzept12}
\end{minipage}
\begin{minipage}{0.45\textwidth}
Das Smartphone von Spieler 1 empfängt die Transaktion noch bevor sie von einem Miner in einen validen Block aufgenommen wurde. Die Wallet Software zeigt die Transaktion erst als unbestätigt an. Sobald sie durch die Aufnahme in Block 15 bestätigt wurde, gilt sie für die Wallet Software als bestätigt.
\end{minipage} 
\section{Umsetzung}

\subsection{Interaktion mit dem Bitcoin Netzwerk}

Möchte man mit dem Bitcoin Netzwerk kommunizieren benötigt man einen Client der das Bitcoin Protokoll implementiert. Dieser kommuniziert dann mit dem Peer-to-Peer Netzwerk und wird dadurch zu einem Knoten (''Node'') des Peer-to-Peer Netzwerks. Man unterscheidet zwischen sogenannten ''Full Nodes'' und ''Light Nodes''.
\subsubsection{Full Node}
Knoten die eigenständig alle Transaktionen und Blöcke auf Gültigkeit mit Hilfe der Konsensregeln prüfen nennt man ''Full Node''. Diese Knoten speichern die gesamte Blockchain und bilden das Rückgrat des Netzwerkes.
Full Nodes stellen in der Regel eine RPC Schnittstelle zur Verfügung. Diese Schnittstelle bietet die Möglichkeit von einer beliebigen Programmiersprache aus mit dem Full node zu interagieren.
Abbildung \ref{fig:btc_core_full_node_architecture}\cite{mastering_bitcoin} zeigt, dass man über die RPC Schnittstelle auf die gespeicherten Daten des Nodes (Blöcke, Blockheader und Adressen) als auch auf die Wallet zugreifen kann.

\begin{figure}[H]
\centering
\includegraphics[width=1\linewidth]{Figures/btc_core_full_node_architecture}
\decoRule
\caption{Bitcoin Core: Full Node Aufbau}
\label{fig:btc_core_full_node_architecture}
\end{figure}

Abbildung \ref{fig:btc_core_full_node_architecture} zeigt außerdem: 
\begin{itemize}
\item Peer Discovery, Peer Datenbank und Connection Manager: Diese kümmern sich um die Kommunikation mit dem Peer-to-Peer Netzwerk.
\item Mempool: Im sogenannten Mempool werden empfangene, unbestätigte Transaktionen im Speicher gehalten.
\item Validation Engine: Diese validiert, ob die empfangenen Blöcke und deren Transaktionen die Konsensregeln einhalten. Falls ja werden die somit bestätigten Transaktionen aus dem Mempool gelöscht und der Block wird an die StorageEngine zur Abspeicherung in der Blockchain Datenbank weitergereicht.
\item Miner: Die Bitcoin Full Node Software enthält einen CPU Miner mit der man mithilfe des Proof-of-Work Algorithmus nach neuen Blöcken suchen kann. Bitcoin Mining ist heutzutage nur noch mit sogenannten ASICs profitabel. ASIC steht für \textbf{A}pplication-\textbf{S}pecific \textbf{I}ntegrated \textbf{C}ircuit. Es handelt sich um Hardware, die auf die Berechnung der SHA256 Hashfunktion spezialisiert ist.
\end{itemize}


\subsubsection{Light Node}
''Light Nodes'' speichern nicht die gesamte Blockchain, sonder in der Regel nur die Blockheader der Blöcke der Blockchain. Beim Mining gehen nur die Daten des Blockheaders in den Blockhash ein. Der Node empfängt Blockheader, prüft ihre Gültigkeit und fügt sie gegebenenfalls in die Headerkette ein. Der Node kann somit eigenständig, d.h. ohne seinen Nachbarn vertrauen zu müssen, die längste Proof-of-Work Kette bilden. Da diese Kette nur aus Headern besteht und keine Transaktionen enthält, kann der Light Node empfangene Transaktionen nicht eigenständig auf ihre Gültigkeit prüfen. Light Nodes verwenden das in \cite{bitcoin_white_paper} beschriebene ''Simplified Payment Verification'' Verfahren zur Prüfung von Transaktionen. ''Light Nodes'' werden daher oft auch ''SPV Client'' genannt. Ein ''SPV Client'' prüft die Gültigkeit einer Transaktion indem er sie an der richtigen Stelle der Headerkette einordnet und dann den passenden Merkel-Branch von einem seiner Nachbarknoten anfragt. Durch diese Zusätzlichen Daten kann er nun wie in Abbildung \ref{fig:spv_chain}\citep{ethereum_white_paper} gezeigt nachprüfen ob der Hash der Transaktion wirklich in den Wurzelknoten des Merkel-Trees mit eingegangen ist.

\begin{figure}[H]
\centering
\includegraphics[width=1\linewidth]{Figures/umsetzung_btc/spv_chain}
\decoRule
\caption{Blockheader Kette}
\label{fig:spv_chain}
\end{figure}

Für den Bitcoin Teil dieser Ausarbeitung ist die Integration mit dem Peer-to-Peer Netzwerk mit Hilfe der in Java geschriebenen BitcoinJ\cite{bitcoinj} Bibliothek umgesetzt.

\subsection{Überblick}

Abbildung \ref{fig:anwendung_aufbau} skizziert die Komponenten der Glücksspielanwendung und wie diese mit ihrer Umgebung kommunizieren. Es gibt zum einen den Server auf dem die Glücksspielanwendung läuft, die Spieler und das Bitcoin Netzwerk.

\begin{figure}[H]
\centering
\includegraphics[width=1\linewidth]{Figures/umsetzung_btc/anwendung_aufbau}
\decoRule
\caption{Glücksspielanwendung Aufbau und Interaktion}
\label{fig:anwendung_aufbau}
\end{figure}

\subsubsection{Glücksspielanwedung}
\begin{itemize}
\item Server: Die Glücksspielanwendung läuft auf einem Server, der über eine Java Virtual Machine (JVM) Laufzeitumgebung verfügt, eine MySQL Datenbank und ein gewöhnliches Dateisystem besitzt.
\item Java Virtual Machine (JVM): Innerhalb der JVM läuft ein sogenannter Application Server, der eine Webanwendung nach außen bereitstellt. Auf diese Webanwendung können die Spieler über das HTTP Protokoll mittels ihres Browsers zugreifen. Die Webanwendung besteht aus mehreren Komponenten.
\item Application Server: Dieser stellt die Applikation bereit. Bei der Umsetzung der Glücksspielanwendung wurde der Open Source Application Server Wildfly\footnote{\url{http://wildfly.org/}} von Red Hat verwendet.
\item GUI: Die Weboberfläche stellt die zentrale Schnittstelle zwischen der Anwendung und dem Spieler da. Diese ist mithilfe des Tapestry\footnote{\url{http://tapestry.apache.org/}} Webframeworks von Apache umgesetzt. Detaillierte Informationen findet man in \citep{tapestry}.
\item Geschäftslogik: Diese Behandelt sowohl die vom Benutzer über die GUI ausgelösten, als auch die vom Bitcoin Netzwerk ausgelösten Events.
\item BitcoinJ: Die Java Bibliothek die zur Kommunikation mit dem Bitcoin Netzwerk verwendet wird.
\end{itemize}

\subsubsection{Spieler}
Die Spieler verfügen über einen Browser und über einen Bitcoin Client.
Mit dem Internetbrowser interagieren sie mit Glücksspielanwendung. Mit dem Bitcoin Client erstellen und empfangen Sie Zahlungen.
\subsubsection{Bitcoin Peer-to-Peer Netzwerk}
Das Peer-to-Peer Netzwerk besteht aus den anderen Teilnehmern des Netzwerks. Dies sind Full-, Light Nodes und Miner. Bei Kryptowährungsnetzwerken unterscheidet man in der Regel zwischen dem Test und Hauptnetzwerk. Den Bitcoins des Testnetzwerks wird kein monetärer Wert zugeschrieben. Das Testnetzwerk dient dazu Software die dem Bitcoin Netzwerk interagieren soll zu testen. Möchte ein Händler Bitcoin in seinen Onlineshop integrieren, kann er so seine Implementierung testen ohne ein finanzielles Risiko einzugehen. 

\subsection{Datenmodel}
Die Grundklasse die die Anwendung verwendet ist die Klasse Pot. Diese repräsentiert ein Spiel und speichert alle relevanten Daten. Sie besteht aus einer Liste von Teilnehmern (Participants). Jeder Teilnehmer hat wie im Konzept beschrieben eine Ein- und Auszahlungsadresse.
\begin{figure}[H]
\centering
\includegraphics[width=1\linewidth]{Figures/umsetzung_btc/btc_datenmodell}
\decoRule
\caption{Java Datenmodel Klassendiagramm}
\label{fig:btc_datenmodell}
\end{figure}


\subsection{Geschäftslogik}

Die Java Klassen der Glücksspielanwendung können, wie in Abbildung \ref{fig:btc_businesslogic} gezeigt, in 3 verschiedene Gruppen unterteilt werden. Das \textbf{Core Module} enthält die Klassen des Datenmodells und ein Interface mit dem die GUI Anwendung interagiert. Das Interface entkoppelt die Anzeigelogik der GUI von der Geschäftslogik der jeweiligen Kryptowährung. Die GUI Komponente bekommt von der Schnittstelle allgemeine Daten und kümmert sich nur um deren Anzeige.
Das \textbf{Bitcoin Module} enthält die gesamte kryptowährungsspezifische Geschäftslogik. \textbf{BitcoinJ} enthält alle Klassen und Interfaces die benötigt werden um mit dem Bitcoin Netzwerk zu interagieren und Daten aus der Blockchain auszulesen.

\begin{figure}[H]
\centering
\includegraphics[width=1\linewidth]{Figures/umsetzung_btc/btc_businesslogic_pdf}
\decoRule
\caption{Java Geschäftslogik Klassendiagramm}
\label{fig:btc_businesslogic}
\end{figure}

\subsubsection{Start der Anwendung}
Beim Kompilieren der Anwendung legt man über einen Konfigurationseintrag fest, ob die Anwendung mit dem Bitcoin Haupt- oder Testnetzwerk interagieren möchte.
Für das Hauptnetzwerk wird die Java Klasse \textbf{BtcMainEjb} verwendet. Für das Testnetwerk wird die Klasse \textbf{BtcTestEjb} verwendet. Beide Klassen verwenden die gleiche Implementierung der abstrakten Oberklasse \code{AbstractBitcoinService}. Diese wiederum implementiert das von der GUI Komponente verwendete Interface \code{CryptoNetworkservice}.

\begin{lstlisting}[basicstyle=\small] %or \tiny or \small or \footnotesize etc.
import javax.ejb.Singleton;
import javax.ejb.Startup;
import org.bitcoinj.params.AbstractBitcoinNetParams;
import org.bitcoinj.params.MainNetParams;
import com.ossel.gamble.bitcoin.services.AbstractBitcoinService;
import com.ossel.gamble.core.data.enums.CryptoNetwork;

@Startup
@Singleton
public class BtcMainEjb extends AbstractBitcoinService {
    @Override
    public CryptoNetwork getCryptoNetwork() {
        return CryptoNetwork.BTC_MAIN;
    }
    @Override
    public AbstractBitcoinNetParams getNetworkParams() {
        return MainNetParams.get();
    }
}
\end{lstlisting}


Die beiden Klassen \code{BtcMainEjb} und \code{BtcTestEjb} sind mit \code{@Startup} und \code{@Singleton} annotiert. Es handelt sich um sogenannte \textbf{E}nterprise \textbf{J}ava \textbf{B}eans. Dies bedeutet, dass der Applikationsserver diese eigenständig managt und genau eine Instanz der Klasse beim Starten der Applikation erzeugt. Beim Start wird dann die mit \code{@PostConstruct} annotierte \code{startSpvNode} Methode aufgerufen und abgearbeitet. Diese konfiguriert und startet das \code{WalletAppKit}, welches die zentrale Klasse zur Interaktion mit der BitcoinJ Bibliothek darstellt.

\begin{lstlisting}[basicstyle=\small]
@PostConstruct
private void startSpvNode() {
    log.info("#### Start Bitcoin SPV Node  ####");
    currentPot = new Pot(2, 100000L);
    File walletDir = CoreUtil.getWalletDirectory();
    NetworkParameters params = getNetworkParams();
    String fileName = "bitcoin-" + params.getPaymentProtocolId();
    bitcoinAppKit = new WalletAppKit(params, walletDir, fileName) {
        @Override
        protected void onSetupCompleted() {
            log.info("#### Bitcoin SPV Node started ####");
        }
    };
    bitcoinAppKit.startAsync();
    waitUntilStarted(bitcoinAppKit);
    newBlockListener = new NewBlockListener(this);
    bitcoinAppKit.chain().addNewBestBlockListener(newBlockListener);
    coinReceivedListener = new CoinsReceivedListener(this);
    bitcoinAppKit.wallet().addCoinsReceivedEventListener(coinReceivedListener);
}
\end{lstlisting}

 In Zeile 4 wird zunächst ein neuer leerer Topf mit 2 Teilnehmern erzeugt. Anschließend wird das \code{WalletAppKit} erzeugt. Dazu bekommt dieses die gewünschten Netzwerkparameter und den Pfad zum Dateisystem in dem \code{BitcoinJ} die Blockchain- und Wallet-Daten speichern soll. Zeile 13 startet den durch das \code{WalletAppKit} repräsentierten SPV Node. Anschließend wird dem \code{WalletAppKit} noch ein \code{NewBlockListener} und ein \code{CoinsReceivedListener} hinzugefügt, um auf neue Blöcke und eingehende Zahlungen zu reagieren.

\subsubsection{GUI Events}

Der folgende Code zeigt die Implementierung der Methoden die von der GUI Komponente aufgerufen werden können.
\begin{lstlisting}
public Pot getCurrentPot() {
    return currentPot.clone();
}
public String getDepositAddress() {
    return bitcoinAppKit.wallet().freshAddress(KeyPurpose.RECEIVE_FUNDS).toString();
}
public String getCurrentBlockHash() {
    return bitcoinAppKit.chain().getChainHead().getHeader().getHash().toString();
}
public int getCurrentBlockHeight() {
    return bitcoinAppKit.chain().getChainHead().getHeight();
}
\end{lstlisting}


Zeile 2 gibt eine Kopie der Topdaten an die GUI Komponente weiter. Da es sich lediglich um eine Kopie des Objektes handelt, ist es ausgeschlossen, dass durch die GUI Komponente der Status des Topfs manipuliert werden kann. Zeile 6 erzeugt eine neue Empfangsadresse. Zeile 9 gibt den Blockhash des neusten Blocks der SPV Blockheader Kette zurück. Zeile 12 gibt die Blocknummer des neusten Blocks der SPV Blockheader Kette zurück.

\subsubsection{Bitcoin Netzwerk Events}

Immer wenn der SVP Node eine Transaktion auf eine vorher mittels der \code{bitcoinAppKit.wallet().freshAddress()} erzeugten Adresse empfängt, wird die \code{onCoinsReceived} Methode aufgerufen.

\begin{lstlisting}[basicstyle=\small]
@Override
public void onCoinsReceived(Wallet wallet, Transaction txn, Coin prevBalance, Coin newBalance) {
  log.debug("Transaction details: " + txn.toString());
  Coin value = txn.getValueSentToMe(wallet);
  Pot currentPot = service.getCurrentPot();
  if (currentPot.getExpectedBettingAmount() > value.getValue()) {
    log.warn("Player did not pay enough.");
    return;
  }
  List<Participant> participants = service.getPossibleParticipants();
  NetworkParameters params = service.getAppKit().params();
  for (TransactionOutput txnOutput : txn.getOutputs()) {
    Address a = txnOutput.getAddressFromP2PKHScript(params);
    String address = a.toString();
    for (Participant participant : participants) {
      String depositAddress = participant.getDepositAddress();
      if (depositAddress.equals(address)) {
        log.info("Received " + value.toFriendlyString() + " coins from " + participant.toString());
        participant.setReceivedAmount(value.getValue());
        String fromAddress = txn.getInput(0).getFromAddress().toString();
        participant.setPayoutAddress(fromAddress);
        wallet.addTransactionConfidenceEventListener(new TxnConfidenceListener(service, txn, participant));
      }
    }
  }
}
\end{lstlisting}




Immer wenn der SVP Node einen neuen besten Block findet, den er vorne an die Header Kette anhängen kann, wird die \code{notifyNewBestBlock} Methode aufgerufen.
\begin{lstlisting}[basicstyle=\small]
@Override
public void notifyNewBestBlock(StoredBlock block) throws VerificationException {
    log.info("New Block height = " + block.getHeight() + " hash = "
            + block.getHeader().getHash().toString());
    List<Pot> unfinishedPots = getUnfinishedPots(service.getClosedPots());
    for (Pot pot : unfinishedPots) {
        long potId = pot.getCreateTime().getTime();
        int payoutBlockHeight = pot.getPayoutBlockHeight();
        if (payoutBlockHeight > block.getHeight()) {
            log.info("Pot[" + potId + "] can not be handled yet.");
        } else if (payoutBlockHeight == block.getHeight()) {
            log.info("Pot[" + potId + "] select temorary Winner.");
            Block tmpPayoutBlock = new ExtendedBlock(block.getHeader().getHash().toString());
            pot.setPayoutBlock(tmpPayoutBlock);
            selectWinner(pot, tmpPayoutBlock);
        } else {
            log.info("Pot[" + potId + "] select final winner.");
            try {
               StoredBlock correctBlock = getPastBlock(payoutBlockHeight, block);
               String payoutBlockHash = correctBlock.getHeader().getHash().toString();
               Block finalPayoutBlock = new ExtendedBlock(payoutBlockHash);
               pot.setPayoutBlock(finalPayoutBlock);
               Participant winner = selectWinner(pot, finalPayoutBlock);
               log.info(winner.getDepositAddress() + " wins pot[" + potId + "].");
               startPayoutThread(pot);
            } catch (BlockStoreException e) {
               log.info("Couldn't select final winner of Pot[" + potId + "]:" + e.getMessage(), e);
            }
        }
    }
}
\end{lstlisting}



Diese Methode iteriert über jeden bereits geschlossenen Topf für den noch keine Auszahlung stattgefunden hat und unterscheidet 3 Fälle:
\begin{enumerate}
\item Der neue Block hat eine Blocknummer, die kleiner ist als die Blocknummer, die den Top entscheidet. In diesem Fall passiert nichts.
\item Der neue Block entscheidet den Topf, da die Blocknummer des Blocks gleich der PayoutHeigth des Topfs ist. Der Gewinner des Topfs wird selektiert, es findet allerdings keine Auszahlung statt. Die finale Auszahlung findet aus Sicherheitsgründen erst im nächsten Fall statt.
\item In diesem Fall gibt es mindestens einen Block, der auf dem Payout-Block des Topfs aufbaut. Ab diesem Zeitpunkt betrachtet die Anwendung den Gewinner als final.\footnote{An dieser Stelle kann man natürlich auch aus Sicherheitsgründen noch mehrere Blöcke abwarten, bevor die Anwendung eine Auszahlung startet.} Daher wird der Gewinner des Topfs überschrieben und die Auszahlung in einem neuen \code{Thread} gestartet.
\end{enumerate}

Die Klasse \code{ExtendedBlock} teilt den Blockhash zur Anzeige in der GUI in die Werte perfix, lastDigit und suffix. Die Variable lastDigit speichert die letzte numerische Stelle des Blockhashs und wird zur Gewinnerauswahl verwendet.

\begin{lstlisting}
public class ExtendedBlock extends Block {

    private String prefix;
    private String suffix;

    public ExtendedBlock(String blockHash) {
        super(blockHash, -1);
        int position = blockHash.length() - 1;
        while (position > 0) {
            char c = blockHash.charAt(position);
            int value = (int) c;
            if (value >= 48 && value <= 57) {// numeric
                this.lastDigit = Integer.parseInt(String.valueOf(c));
                break;
            }
            position--;
        }
        this.prefix = blockHash.substring(0, position);
        this.suffix = blockHash.substring(position + 1, blockHash.length());
    }
}
\end{lstlisting}







\subsubsection{Auszahlungen}
Auszahlung werden in einem eigenen Thread abgehandelt. Die Klasse \code{PayoutThread} ruft dazu die \code{payout} Methode auf.
\begin{lstlisting}[basicstyle=\small]
private void payout(Pot pot) throws InsufficientMoneyException, InterruptedException, ExecutionException {
    if (pot.isPayoutStarted()) {
        log.error("Payout already started: " + pot.getPayoutTxnId() + " - " + pot.getPayoutError());
    } else {
        pot.setPayoutStarted(true);
        Address winnerAddress = new Address(bitcoinService.getNetworkParams(), pot.getWinner().getPayoutAddress());
        Coin potValue = Coin.SATOSHI.multiply(pot.getParticipants().size() * pot.getExpectedBettingamount());
        Wallet.SendResult result = bitcoinService.getAppKit().wallet()
        .sendCoins(bitcoinService.getAppKit().peerGroup(), winnerAddress, potValue);
        String txnId = result.tx.getHash().toString();
        pot.setPayoutTxnId(txnId);
        log.info("Payout TXN ID = " + txnId);
        Transaction transaction = result.broadcastComplete.get();
        log(transaction);
    }
}
\end{lstlisting}


Die Methode prüft erst, dass die Auszahlung noch nicht gestartet wurde. Ist dies der Fall, wird der auszuzahlende Betrag berechnet und an die Adresse des Gewinners überwiesen. Anschließen wird die ID der Auszahlungstransaktion in den Topf geschrieben. Während der Auszahlung können von BitcoinJ 3 verschiedene Fehler auftreten:
\begin{enumerate}
\item InsufficientMoneyException: Falls die von der Wallet verwalteten Adressen nicht genug Bitcoin für die Auszahlung besitzen.
\item InterruptedException: Falls der Java Thread unterbrochen wird.
\item ExecutionException: Falls es zu einem unerwarteten Fehler bei der Ausführung kommt.
\end{enumerate}
Sollte es bei der Auszahlung ein Problem geben, wird die Exception gefangen und abgespeichert.

\subsection{Grafische Benutzeroberfläche}\label{ssec:btc_gui}

Die Graphische Oberfläche der Anwendung ist mit dem Tapestry Framework von Apache realisiert. Da die GUI Komponente nur die Daten visualisiert, wird an dieser Stelle auf eine genauere Betrachtung verzichtet und lediglich die Benutzeroberfläche gezeigt.


\begin{figure}[H]
\centering
\includegraphics[width=1\linewidth]{Figures/btc_gui/pot_open2}
\decoRule
\caption{Leerer Topf}
\label{fig:pot_open2}
\end{figure}
Abbildung \ref{fig:pot_open2} zeigt einen Topf mit 2 freien Plätzen. Um dem Spiel beizutreten, muss der Spieler den Betrag von 0,001 Bitcoin an die angezeigte Adresse senden. Der angezeigte QR-Code erleichtert dem Spieler die Übertragung dieser Daten in das Überweisungsformular seines Smartphone Wallets. Das \textbf{B}itcoin \textbf{I}mprovement \textbf{P}roposal Nummer 21\citep{bip21} legt fest, in welchem Format diese Daten kodiert werden müssen, damit beliebige Bitcoin Clients diese korrekt auslesen können. 
Folgende Daten sind in dem QR Code enthalten: ''bitcoin:mws1YS...Ae5?amount=0.01\&message=gambling''

\begin{figure}[H]
\centering
\includegraphics[width=1\linewidth]{Figures/btc_gui/handy_send}
\decoRule
\caption{Zahlung mit Smartphone}
\label{fig:handy_send}
\end{figure}
Abbildung \ref{fig:handy_send} zeigt das Bitcoin CoPay Wallet\footnote{\url{https://copay.io/}}. Dieses erlaubt es Zahlungen an das Bitcoin Testnetz zu senden. Nachdem der Benutzer den QR-Code der Glücksspielanwendung mit seinem Smartphone abgescannt hat, erscheint sowohl der Betrag als auch die Empfangsadresse vor-ausgefüllt im Überweisungsformular. Die Wallet berechnet automatisch eine passende Transaktionsgebühr. Der Spieler prüft lediglich die Adresse und den Betrag und  autorisiert anschließend die Zahlung.

\begin{figure}[H]
\centering
\includegraphics[width=1\linewidth]{Figures/btc_gui/pot_closed}
\decoRule
\caption{Topf geschlossen}
\label{fig:pot_closed}
\end{figure}
Nachdem die Zahlung von der Glücksspielanwendung erkannt wurde und ein weiterer Spieler in den Topf eingezahlt hat, ist der Topf voll. Nun wartet die Anwendung auf den Payout Block, der den Gewinner des Topfs festlegt. Da der Blockhash noch nicht feststeht, zeigt die Anwendung die Animation eines sich sehr schnell wechselnden Block Hash, der mehrere Fragezeichen enthält, an. Dies ist in Abbildung \ref{fig:pot_closed} zu sehen.

\begin{figure}[H]
\centering
\includegraphics[width=1\linewidth]{Figures/btc_gui/pot_settled}
\decoRule
\caption{Gewinner ermittelt}
\label{fig:pot_settled}
\end{figure}
Sobald der entscheidende Block empfangen wird, wird der voraussichtliche Gewinner des Topfs durch ein rotes Rechteck markiert. Nun wartet die Anwendung bis ein weiterer Block gefunden wird, bevor sie die Auszahlung an den Gewinner startet. 

\begin{figure}[H]
\centering
\includegraphics[width=1\linewidth]{Figures/btc_gui/pot_payout_finished}
\decoRule
\caption{Auszahlung beendet}
\label{fig:pot_settled}
\end{figure}

Klickt der Benutzer auf den \emph{Payout triggered} Link, gelangt er zu einem Blockexplorer. Dieser zeigt ihm die Transaktionsdetails an.  
\section{Evaluation}
\subsection{Prüfung der Anforderungen}%Anforderungserfüllung}

Dieser Abschnitt behandelt in wie weit das beschriebene Konzept die in Kapitel 1 aufgelisteten Anforderungen erfüllt. Die jeweilige Anforderung wird zunächst wiederholt und anschließend genauer untersucht.

\subsubsection{Transparente Einzahlungen}
Die Einzahlung jedes Endnutzers ist für jeden anderen Endnutzer nachprüfbar.\\\\
Diese Anforderung ist erfüllt, da jede Transaktion in der lokalen Datenbank jedes Peet-to-Peer Netzwerkteilnehmers aufgezeichnet wird. 
Auf der Webseite \cite{blockchain_info} kann man die Bitcoin Blockchain mithilfe eines sogenannten Blockchain-Exploers durchsuchen. Mit diesem Werkzeug kann man die Blöcke und die darin enthaltenen Transaktionen untersuchen. Nutzt man den Explorer einer Drittpartei, muss man darauf vertrauen, dass dieser auch den ''wahren'' Status der Blockchain anzeigt. Um dieses Risiko zu vermeiden kann jeder Teilnehmer mithilfe eines eigenen Bitcoin Full Node am Netzwerk teilnehmen. Dieser speichert die gesamte Blockchain und prüft alle Transaktionen und Blöcke gegen die Konsensregeln.\\
Der Bitcoin Full Node stellt eine API bereit, über die man den aktuellen Status der Blockchain abfragen kann. Der Befehl \textbf{getblockchaininfo} liefert den aktuellen Zustand der Blockchain zurück. Dieser beinhaltet die Blocknummer des neusten Blocks und dessen Blockhash. Der Befehl \textbf{gettransaction} gefolgt von der Transaktions-ID liefert Details über eine Transaktion. Die Webseite \cite{btc_api} dokumentiert diese Schnittstelle detailliert. 

\subsubsection{Gewinnerauswahl durch Zufallsfaktor}
Die Auswahl des Gewinners ist von einem zufälligen Faktor abhängig, auf den weder die Anwendung noch die Endnutzer einen Einfluss haben.\\\\
Diese Anforderung wird nur bedingt erfüllt, da ein Teilnehmer des Peer-to-Peer Netzwerks sowohl ein Spieler als auch ein Miner sein kann. Ist dies der Fall besteht die Möglichkeit, dass der Miner einen validen Blockhash verwirft, sobald er merkt, dass er durch diesen Blockhash nicht zum Gewinner des Geldtopfes wird. Verwirft der Teilnehmer einen Blockhash, riskiert er den dadurch ausgeschütteten Blockreward. Ein solcher Angriff ist für einen Miner nur rentabel, falls die Spieleinsätze des Geldtopfes den Blockreward um ein vielfaches übersteigen.
Betrachten wir dazu das Bitcoin Netzwerk Anfang Februar 2018. Der Preis pro Bitcoin beträgt 8000 Euro. Der Mining Reward liegt bei 12,5 Bitcoin pro Block. Für das Lösen eines gültigen Blocks erhält ein Miner somit 100000 Euro. Angenommen ein Miner besitzt 20 Prozent der Hashrate des gesamten Bitcoin-Netzwerks und nimmt an einem Topf mit 2 Personen teil. Dies bedeutet, dass sowohl der Miner als auch der andere Teilnehmer eine Gewinnwahrscheinlichkeit von 0,5 für einen zufälligen Blockhash haben.
Da der Miner eine Hashrate von 20 Prozent hat, liegt die Wahrscheinlichkeit das der Miner den nächsten Block findet bei 0,2. Falls er den gefundene Blockhash verwirft, da er durch diesen seinen Glücksspieleinsatz verlieren würde, muss er es schaffen vor einem anderen Miner einen weiteren Blockhash zu berechnen. Ansonsten verliert er den Blockreward. Die Wahrscheinlichkeit das der Miner zwei gültige Blockhashs hintereinander findet, liegt bei 0,2 * 0,2 = 0,04 und ist somit verschwindend gering.

\subsubsection{Nachprüfbarkeit des Zufallsfaktor}
Jeder Endnutzer kann die Echtheit des zufälligen Faktors eigenständig nachprüfen.\\\\
Da das Verfahren der Gewinnerauswahl im Vorhinein festgelegt ist und die Reihenfolge der Einzahlungstransaktionen in der Blockchain festgeschrieben steht, kann jeder Teilnehmer die Berechnung des Gewinners eigenständig nachvollziehen. Der Blockhash der die Grundlage für die Gewinnerauswahl liefert kann durch die Verwendung eines Bockchain-Explorers oder eines Full Nodes nachgeprüft werden. 
\subsubsection{Transparente Auszahlungen}
Die Auszahlung an den Gewinner muss transparent und somit für jeden Endnutzer nachprüfbar sein.\\\\
Genau wie die Einzahlungen ist auch die Auszahlung für jeden Spieler mithilfe eines Blockchain-Explorers oder eines Bitcoin Full Nodes möglich. Jeder Teilnehmer kann somit für alle bereits abgeschlossenen Spiele nachprüfen ob die Anwendung sich korrekt verhalten und eine Auszahlung getätigt hat. 
\subsubsection{Fairheit des Spiels}
Jeder Endnutzer besitzt die gleiche Gewinnwahrscheinlichkeit und niemand wird benachteiligt.\\\\
Die Zuordnung der Spieler auf die Gewinnzahlen ist durch die Reihenfolge der Transaktionen in der Blockchain festgeschrieben. Eine nachträgliche Veränderung dieser Reihenfolge ist weder durch die Nutzer, noch durch die Glücksspielanwendung möglich.\footnote{Eine Veränderung der Reihenfolge ist nur durch einen sogenannten Blockchain-Fork möglich. Kapitel \ref{Chapter6} erörtert welche Auswirkungen dies auf die Glücksspielanwendung hat.}

Damit keiner der Spieler einen Vorteil hat, muss jeder Topf-Platz die gleiche Gewinnwahrscheinlichkeit haben.
Dies ist gegeben, falls a) jeder Teilnehmer die gleiche Anzahl Gewinnzahlen zugeordnet bekommt und b) falls die möglichen Blockhash-Werte für die Gewinnerauswahl gleichverteilt sind.\\

a) Die Gewinnerauswahl kann entweder wie im Konzept beschrieben durch den gesamten Blockhash-Wert oder wie im Beispiel auf Basis der letzten Blockziffer vorgenommen werden.
Beide Methoden haben Vor- und Nachteile.\\ Variante eins erlaubt beliebige Topfgrößen, ist dafür aber schwieriger für den Endnutzer zu verifizieren. Die Verifizierung erfordert die Konvertierung des Blockhashs ins Dezimalsystem und eine Modulo-Rechnung einer sehr große Zahl.\\ Variante zwei ist dagegen leicht zu verifizieren, erlaubt allerdings nur die Topfgrößen zwei, fünf und zehn. Bei der Topfgröße von zwei sind beiden Spielern fünf Gewinnzahlen zugeordnet. Bei der Topfgröße von fünf besitzt jeder Teilnehmer genau 2 Gewinnzahlen. Bei einer Topfgröße von zehn wird jedem Teilnehmer genau eine Gewinnzahl zugeordnet. 
Nimmt man hingegen eine Topfgröße von 1,3,4,6,7,8 und 9 führt dies dazu das manche Teilnehmer eine signifikant höhere Gewinnchance haben.
Bei der Topfgröße von 3 sind die Gewinnzahlen durch die Modulo-Funktion folgendermaßen verteilt:
\begin{itemize}
\item Spieler 1 hat die Gewinnzahlen 0, 3 und 9.
\item Spieler 2 hat die Gewinnzahlen 1 und 4.
\item Spieler 3 hat die Gewinnzahlen 2, 5 und 8.
\end{itemize}
Somit haben Spieler 1 und 3 eine Gewinnwahrscheinlichkeit von $\frac{3}{10}$, Spieler 2 hingegen nur eine Gewinnwahrscheinlichkeit von $\frac{2}{10}$.

Es kommt also vor, dass eine Teilmenge der Spieler genau eine Gewinnzahl mehr als der Rest der Teilnehmer hat.
Nimmt man den gesamten Blockhash zur Gewinnerauswahl ist die dadurch entstehende Ungerechtigkeit verschwindend gering und kann vernachlässigt werden. Dies ist der Fall, da die aus dem Blockhash resultierende Dezimalzahl in der Praxis sehr groß ist und jeder Spieler somit mehrere Millionen von Gewinnzahlen hat.\\

b) Der Blockhash eines Blocks wird durch die verwendete kryptographische Hashfunktion der Kryptowährung festgelegt. Die Verteilung der Werte ist somit von der verwendeten kryptographische Hashfunktion abhängig.\\
Eine kryptografische Hashfunktion ist eine stark kollisionsresistente Einweg-Hashfunktion.\\
Eine Hashfunktion h heißt
\begin{itemize}
\item Einwegfunktion genau dann, wenn es schwierig ist, zu gegebenem $_{Y0}$ ein $_{X0}$ zu finden mit $h(x_{0}) = y_{0}$.
\item schwach kollisionsresistent genau dann, wenn es schwierig ist, zu einem gegebenen $x$ ein $x'$ zu finden mit $h(x) = h(x')$.
\item stark kollisionsresistent genau dann, wenn es schwierig ist, $x$ und $x'$ zu finden
mit $x \neq xx$ und $h(x) = h(xx)$.
\end{itemize}Die Eigenschaften der starken Kollisionsresistenz und der Einwegfunktion sagen nichts über die Verteilung der ausgegebenen Werte aus. Bei der Auswahl der Kryptowährung muss also gesondert auf die Verteilung der verwendete Hashfunktion geachtet werden. Sollte die verwendete kryptographische Hashfunktion keine Gleichverteilung liefern, kann der Blockhash dennoch den nötigen Zufall liefern indem dieser mit einer geeigneten Hashfunktion erneut gehasht wird.\\\\
Bitcoin verwendet die kryptographische Hashfunktion SHA256.
Die folgende Monte-Carlo-Simulation zeigt, dass die Resultate der SHA256 gleichverteilt sind.
\begin{verbatim}
h=SHA256 n=1000000
for i 1 -> n
    hash = h(i);
    result[lastDigit(hash)]++
\end{verbatim}
\begin{minipage}{0.5\textwidth}
\begin{verbatim}
Ausgabe:
result[0] =  99765
result[1] = 100488
result[2] =  99913
result[3] = 100745
result[4] = 100272
result[5] =  99649
result[6] =  99430
result[7] =  99788
result[8] =  99666
result[9] = 100284
\end{verbatim}
\end{minipage}
\begin{minipage}{0.5\textwidth}
\includegraphics[width=\textwidth]{Figures/verteilung_sha256}
\centering
\decoRule
\captionof{figure}{\label{fig:verteilung_sha256}Verteilung der SHA256 Hashfunktion}
\label{fig:verteilung_sha256}
\end{minipage}

\pdfcomment{Hier dann noch die Verteilung der echten Bitcoin Blockchain angeben.}
%Somit ist das Spiel nur unter der Annahme fair, dass die Zahl zur Gewinnerauswahl gleichverteilt ist.

\subsection{Betrugsmöglichkeiten}

Dieser Abschnitt betrachtet in wie weit die Glücksspielanwendung potentielle Spieler betrügen kann, sollte sie gehackt und zu ausschließlich diesem Zweck modifiziert werden.
Die Anwendung hat die volle Kontrolle darüber welche Ausgabe sie dem Benutzer anzeigt. Sie hat allerdings keine Kontrolle über den Status der Blockchain. 

Die Anwendung könnte beispielsweise anzeigen, dass der Topf nach der Einzahlung durch einen Spieler immer noch leer ist. Die Transaktion auf die Einzahlungsadresse existiert dann zwar in der Blockchain allerdings reagiert die Anwendung nicht entsprechend. Dies hat zur Folge, dass jeder Spieler der eine Einzahlung tätigt, sein Geld verliert. Allerdings merkt der Benutzer dies und kann somit eine weitere Verwendung der Anwendung unterlassen. Ein solch plumper Manipulationsversuch fällt somit direkt auf.

Ein weitere Betrugsmöglichkeit ist, dass die Glücksspielanwendung sich bis zur Gewinnerauswahl korrekt verhält, dann allerdings keine Auszahlung tätigt. Alle einzahlenden Spieler merken den Betrug, verlieren aber dennoch ihr Geld.

Bei beiden vorgestellten Betrugsversuchen fällt der Betrug immer mindestens einem Spieler auf. Die Verwendung einer solchen Anwendung macht nur Sinn, falls man den Betreiber des Services kennt und diesen im Zweifel juristisch haftbar machen kann.\\
Das folgende Kapitel betrachtet wie sogenannte ''Smart Contract''s dieses Problem lösen. Ein Smart Contract erlauben es die Geschäftslogik der Glücksspielanwendung in die Blockchain zu schreiben. Die Geschäftslogik wird somit nicht mehr von der Anwendung, sondern von jedem Teilnehmer des Peer-to-Peer Netzwerks ausgeführt. 


\subsection{Angriff durch Miner}

\pdfcomment{Überlegungen wie man es nicht machen kann:Blockhash 100 mal hashen bringt nicht. Miner findet Hash und macht erst danach die Arbeit. Die 5 letzten Blockhashs mit einbeziehen bringt auch nichts, weil es dann nur auf den 5. letzten Block ankommt}

\subsection{Blockchain Mining Varianz}

\pdfcomment{Hier gibt es einen Vorschlag, wie man die erhoffte Blocktime öffters erreichen kann.
Bobtail: A Proof-of-Work Target that Reduces Blockchain Mining Variance (Brian N. Levine)
https://stanford2017.scalingbitcoin.org/presentations}

\subsection{Auszahlungstransaktion}
Die Glücksspielanwendung erzeugt für jede Einzahlung eine eigene Einzahlungsadresse statt für jeden Benutzer die gleiche Adresse zu verwenden. Dies hat den Vorteil, dass die Anwendung dem Benutzer anzeigen kann, dass sein Bitcoin Einzahlung eingegangen ist. Der Nachteil ist, dass dadurch die Größe der Auszahlungstransaktion steigt und man somit eine höhere Transaktionsgebühr zahlen muss. Im folgenden wird die Auszahlungstransaktion des Beispiels aus \ref{ssec:btc_gui} betrachtet.

\begin{figure}[H]
\centering
\includegraphics[width=1\linewidth]{Figures/btc_gui/btc_txn}
\decoRule
\caption{Auszahlungstransaktion Details}
\label{fig:btc_txn}
\end{figure}

\label{fig:btc_txn} zeigt in welchen Block die Transaktion aufgenommen wurde. Den Status, den Wert, die Transaktionsgebühr und die Größe der Transaktion.
\footnote{Momentan zahlt die Glücksspielanwendung die Auszahlungstransaktionsgebühr aus eigener Tasche. Eigentlich müsste die Transaktionsgebühr von dem Gewinnbetrag abgezogen werden.}

\begin{figure}[H]
\centering
\includegraphics[width=1\linewidth]{Figures/btc_gui/btc_txn_input_output}
\decoRule
\caption{Auszahlungstransaktion Inputs und Outputs}
\label{fig:btc_txn_input_output}
\end{figure}


\label{fig:btc_txn_input_output} zeigt welche Inputs und Outputs für die Transaktion verwendet wurden. Output Adresse 0 gehört dem Wallet der Glücksspielanwendung und stellt die Wechselgeldadresse dar.


\begin{figure}[H]
\centering
\includegraphics[width=1\linewidth]{Figures/btc_gui/btc_txn_input_output_scripts}
\decoRule
\caption{Auszahlungstransaktion Skripts}
\label{fig:btc_txn_input_output_scripts}
\end{figure}

Da die Anwendung für jeden Benutzer eine eigene Adresse generiert, muss die Anwendung in der Auszahlungstransaktion für jede Input Adresse eine gültige Signatur angeben.\label{fig:btc_txn_input_output_scripts} zeigt, dass die Transaktion dadurch wesentlich größer wird. Hier könnte in Zukunft die Verwendung sogenannter Schnorr Multi-Signaturen aushelfen. Durch diese lassen sich alle Signaturen der Inputs durch eine einzige Signatur ersetzen. 


\chapter{Zweiter Ansatz: Ethereum} % Main chapter title

\label{eth} % For referencing the chapter elsewhere, use \ref{eth} 

\section{Grundlagen}
Ethereum ist genau wie Bitcoin ein Peer-to-Peer Netzwerk Protokoll, dass auf einer öffentlichen Blockchain basiert und einen Systemzustand verwaltet. Es handelt sich im Gegensatz zu Bitcoin um eine generalisierte Blockchain die nicht nur Finanztransaktionen speichern kann, sondern sogenannte Smart Contracts. Ethereum sieht sich als eine Open Source Plattform, die es ermöglicht dezentrale Applikationen zu entwickeln und bereitzustellen.

\subsection{Smart Contracts}
Ethereum ermöglicht es Geschäftsprozesse in Form von Smart Contracts zu programmieren und von einem globalen dezentralen Netzwerk ausführen zu lassen. Ein Contract ist ein Programm bestehend aus einer Reihe von Anweisungen, die ausgeführt werden, sobald das Programm eine Mitteilung in Form einer Transaktion erhält. Contracts haben die Möglichkeit Daten aus der Blockchain auszulesen und auf ihr Daten zu speichern. Smart Contracts können sowohl von Menschen als auch von anderen Contracts ausgelöst werden. Contracts können daher mit anderen Contracts über die vom Programmierer festgelegte Schnittstelle interagieren.
\subsection{Ethereum Accounts}
Um Ethereum nutzen zu können braucht man einen Account. Ethereum Accounts haben:\\
\begin{itemize}
\item Eine 20 Byte lange Adresse,
\item Einen Kontostand in Ether,
\item Contract Code der durch den Account verwaltet wird,
\item Speicherplatz,
\end{itemize} In Ethereum gibt es 2 Arten von Accounts. Die erste Art ist eine Adresse die durch den Besitzer des privaten Schlüssels kontrolliert wird. Diese Art von Account ist vergleichbar mit einer Bitcoin Adresse, die man durch eine Bitcoin-Wallet kontrolliert. Die zweite Art ist ein Contract Account, der durch den Contract Code kontrolliert wird. Der Contract Code verwaltet eigenständig den Kontostand des Accounts und verhält sich genauso wie der Programmierer es festgelegt hat.
\subsection{Transaktionen}
Interaktionen mit dem Ethereum Netzwerk finden in Form von sogenannten Transaktionen statt. Transaktionen beinhalten:
\begin{itemize}
\item eine Empfangsadresse.
\item eine Anzahl an Ether die versandt wird.
\item Daten (Bytes), die vom Contract ausgelesen werden können.
\end{itemize}

Es gibt zwei verschiedene Arten von Transaktion. Sie unterscheiden sich durch die in der Transaktion angegebene Empfangsadresse. 
Wird die Empfangsadresse durch einen privaten Schlüssel kontrolliert, handelt es sich lediglich um eine Finanztransaktion, die Ether vom Sender zum Empfänger Account transferiert. Anderenfalls handelt es sich um eine Transaktion, die den Contract Code der Empfangsadresse ausführt. Welche Funktion ausgeführt werden soll, wird vom Sender im Datenfeld der Transaktion kodiert. Jeder Netzwerkknoten arbeitet jede einzelne Transaktion auf der Blockchain ab und speichert den gesamten resultierenden Status. 
\subsection{Ethereum Virtual Machine}
In ihr wird der Code der Contracts ausgeführt. Sie arbeitet die Anweisungen des Contracts der Reihe nach ab und bricht ab, falls in der Transaktion nicht genügend Ether für die Ausführung des Codes bezahlt wurden.
\subsection{Ether}. 
Ist die Kryptowährung des Ethereum Netzwerks. Sie wird benutzt um für Transaktionsgebühren und die Ausführung von Contract Code zu bezahlen. Ether entsteht ähnlich wie bei Bitcoin durch Mining.
\subsection{Ethereum Client}. 
Einen Ethereum Client wird benötigt. um am Netzwerk teilnehmen zu können. Vor der Verwendung muss dieser sich erst mit dem Netzwerk synchronisieren. Bei der Synchronisation werden die neuen Transaktionen heruntergeladen und überprüft. Weit verbreitete Ethereum Clients sind Ethereum Wallet und Mist.

\section{Konzept} \label{eth_konzept}

Der Ablauf des Spiels ist mit dem Ablauf des Spiels des Bitcoin Kapitels nahezu identisch. Die Unterschiede sind, dass das gesamte Spiel vom Nutzer initiiert wird und die Glücksspielanwendung ausschließlich den Spielstatus anzeigt. Dies führt dazu, dass die Spieler keinerlei Vertrauen in die Glücksspielanwendung setzen müssen. Statt aufgrund leichter Überprüfbarkeit die letzte numerische Stelle des Blockhashs für die Gewinnerauswahl zu nutzen, kann bei Ethereum der gesamten Blockhash zur Gewinnerauswahl genutzt werden. Der Spieler kann aufgrund der Konsensregeln darauf vertrauen, dass das Ethereum Netzwerk die Ausführung des Contract Codes korrekt durchführt und die modulo Funktion zur Gewinnerauswahl korrekt berechnet. Die Nutzung des gesamten Blockhashs als Zufallsquelle ermöglicht Töpfe beliebiger Größe. 
Der Ablauf des Spiels lässt sich durch den folgenden Zustandsautomat beschreiben.

\begin{figure}[H]
\centering
\includegraphics[width=1\linewidth]{Figures/umsetzung_eth/smart_contract_automat_idea}
\decoRule
\caption{Smart Contract Automat}
\label{fig:smart_contract_automat_idea}
\end{figure}

Durch den Aufruf der \code{deploy} Funktion wird der Smart Contract in einer Transaktion an das Netzwerk gesendet und innerhalb eines Blocks in die Blockchain aufgenommen. Im Smart Contract sind der Einzahlungsbetrag und die Anzahl Teilnehmer \code{n} fest definiert. Ab diesem Moment kann der Code des Smart Contract nicht mehr verändert werden. Im initialen Zustand ist der Topf leer, da noch kein Spieler eingezahlt hat. Nun können genau \code{n} Einzahlungen von Spielern durch den Aufruf der \code{deposit} Funktion getätigt werden. Dazu benötigen die Spieler lediglich einen Ethereum Client, ausreichend Ether\footnote{Einzahlungsbetrags plus Transaktionskosten und optimaler Weise genug Ether für das Bezahlen des Aufrufs der \code{payout} Funktion.} und die Adresse des Smart Contracts. Ab dem Zeitpunkt an dem der Smart Contract die letzte Einzahlungstransaktion empfängt, wird der Topf geschlossen und die Blocknummer für die Gewinnerauswahl festgelegt. Der Block, der den Gewinner des Topfs entscheidet, muss in der Zukunft liegen und darf nicht vorher bekannt sein, da sonst Betrugsmöglichkeiten entstehen. Zum Zeitpunkt der letzten Einzahlungstransaktion ist es nicht möglich aus dem Smart Contract Code heraus auf den Blockhash des Blocks zuzugreifen in dem sich die letzte Einzahlungstransaktion befindet. Dies liegt daran, dass die Miner das Resultat der Zustandsveränderung aller Transaktionen des Blockes in den Blockheader schreiben müssen und erst anschließend den Blockhash berechnen. Die Transaktionen, die den Contract Code ausführen, können somit nicht auf den Blockhash zugreifen, da dieser zum Zeitpunkt der Codeausführung noch nicht feststeht. Es ist somit unumgänglich nach der letzten Einzahlungstransaktion eine Funktion aufzurufen, die den Gewinner auswählt, die Auszahlung startet und den Topf anschließend für ein neues Spiel wieder öffnet. Dies ist die in Abbildung \ref{fig:smart_contract_automat_idea} gezeigte \code{payout} Funktion.  
\section{Umsetzung}

\subsection{Überblick}


In dieser Masterarbeit wird unsere Applikation über die RPC Schnittstelle eines Full Nodes an das Ethereum-Netzwerk angebunden. Dies ist in Abbildung \ref{fig:ethereum_integration} verdeutlicht.
 
\begin{figure}
\centering
\includegraphics[width=1\linewidth]{Figures/ethereum_integration}
\decoRule
\caption{Ethereum: Netzwerk Integration}
\label{fig:ethereum_integration}
\end{figure}


\subsection{Smart Contract}
Die folgenden Codestücke beschreiben den TrustlessGambling Smart Contracts in der Sprache Solidity.
\subsubsection{Datenmodell}

\begin{lstlisting}[basicstyle=\small]
pragma solidity ^0.4.0;
contract TrustlessGambling {
    // constants
    uint8 public constant NBR_OF_SLOTS =3;
    uint public constant EXPECTED_POT_AMOUNT=1000;// wei
    uint8 public constant PAYOUT_BLOCK_OFFSET =1;    
    // pot values
    uint public nbrOfParticipants;
    address[NBR_OF_SLOTS] public depositAddresses;
    address[NBR_OF_SLOTS] public payoutAddresses;
    uint public closingBlockNumber;
    uint public payoutBlockNumber;
    bytes32 public payoutBlockHash;
    uint public winner; // 0 -> NBR_OF_SLOTS-1
    bool public potClosed;
    uint public nbrOfMissedPayouts;
    // constructor
    function TrustlessGambling() public {
        nbrOfParticipants = 0;
        potClosed = false;
        nbrOfMissedPayouts = 0;
    }
}
\end{lstlisting}



\subsubsection{Einzahlungen}

\begin{lstlisting}
function deposit() payable public {
    deposit(msg.sender);
}
function deposit(address _payout) payable public {
    assert(msg.value == EXPECTED_POT_AMOUNT);
    assert(!potClosed);
    depositAddresses[nbrOfParticipants] = msg.sender;
    payoutAddresses[nbrOfParticipants] = _payout;
    nbrOfParticipants++;
    if (nbrOfParticipants == NBR_OF_SLOTS){
        closingBlockNumber = block.number;
        payoutBlockNumber = closingBlockNumber + PAYOUT_BLOCK_OFFSET;
        potClosed = true;
    }
}
\end{lstlisting}



\subsubsection{Auszahlungen}

\input{CodeSnippets/SmartContract/auszahlungen.tex}

\subsubsection{Einschränkungen}

Da man bei Ethereum im Conrtact Code nur auf die 256 letzten Blockheader zugreifen kann, unterscheidet sich der Smart Contract leicht von dem im Konzept beschriebenen.

\begin{figure}[H]
\centering
\includegraphics[width=1\linewidth]{Figures/umsetzung_eth/smart_contract_automat}
\decoRule
\caption{Smart Contract Umsetzung}
\label{fig:smart_contract_automat}
\end{figure}

\subsection{Smart Contract Bereitstellung}
Nachdem man den Smart Contract programmiert hat, muss man ihn zu Bytecode kompilieren und anschließen in einer Transaktion an das Ethereum Netzwerk senden. Dazu kann man die von Web3J bereitgestellten Comandline Tool nutzen. Dieses Tool hilft bei der Generierung einer Wallet und erlaubt es aus dem Contract Code eine Java Klasse zu generieren. Dieser Klasse ermöglicht die Interaktion mit dem Smart Contract. 

\begin{lstlisting}[basicstyle=\small]
public void createContract() throws Exception {
  String WALLET_FILENAME = "ethereum.json";
  String WALLET_PASSWORD = "changeit";
  long GAS_LIMIT = 1000000;
  ClassLoader classLoader = getClass().getClassLoader();
  File walletFile = new File(classLoader.getResource(WALLET_FILENAME).getFile());
  Credentials credentials = WalletUtils.loadCredentials(WALLET_PASSWORD, walletFile.getAbsolutePath());
  System.out.println("Account address = " + credentials.getAddress());
  Web3j web3j = Web3j.build(new InfuraHttpService("https://rinkeby.infura.io/" + UserConfiguration.API_KEY));
  BigInteger currentGasPrice = web3j.ethGasPrice().send().getGasPrice();
  TrustlessGambling contract = TrustlessGambling.deploy(web3j, credentials, currentGasPrice, BigInteger.valueOf(GAS_LIMIT)).send();
  String status = contract.getTransactionReceipt().get().getStatus();
  if ("0x1".equals(status)) {
    String address = contract.getContractAddress();
    System.out.println("Contract address = " + address);
    System.out.println("TXN hash = " + contract.getTransactionReceipt().get().getTransactionHash());
    System.out.println("Gas used = " + contract.getTransactionReceipt().get().getGasUsed());
  } else {
    System.out.println("Smart contract could not be deployed.");
  }
}
\end{lstlisting}
Die Ausführung des oben gezeigten Java Codes führt zu der folgenden Ausgabe:

\begin{lstlisting}
Account address = 0x2201f3919589b519135ce977cc0906c9481069b2
Contract address = 0x25c3136145fbd7f3b9217e58e2fabe3eb1928705
TXN hash = 0x06dce3c460b4caa595c5cc0f81ac78e7c70eeb1e89d3e0e6a017ea88e60dbce1
Gas used = 825846
\end{lstlisting}

In einem Blockchain Explorer kann man die Details der vom Full Node erstellten Transaktion\footnote{\url{https://rinkeby.etherscan.io/tx/0x06dce3c460b4caa595c5cc0f81ac78e7c70eeb1e89d3e0e6a017ea88e60dbce1}} und den kompilierten Contract Code\footnote{\url{https://rinkeby.etherscan.io/address/0x25c3136145fbd7f3b9217e58e2fabe3eb1928705\#code}} anschauen.
Alternativ zu Web3J lässt sich der Contract Code mithilfe eines Online Compilers \footnote{\url{https://ethereum.github.io/browser-solidity}} kompilieren und mithilfe des Ethereum Clients namens Mist\footnote{\url{https://github.com/ethereum/mist}} veröffentlichen.

\subsection{Geschäftslogik Webanwendung}

Dieses Kapitel zeigt, wie man in Etherem über einen Full Node mit dem Netzwerk interagieren kann. Die Webanwendung zeigt lediglich lediglich den aktuellen Zustand des Smart Contracts an. Die gesamte Geschäftslogik des Smart Contracts wird vom Etherem Netzwerk ausgeführt. Sollte die Webanwendung aufgrund  technischer Fehler ausfallen, hat dies keinerlei Auswirkung auf das eigentliche Spiel.

TODO 

\iffalse
\begin{enumerate}
\item Es gibt sowohl test als auch mainnet. Unterscheiden sih nur leicht durch protokoll und port. Teil sehen adressen anders aus. Bei ethereum creiert man sich ein wallet und die adresse ist für beide netzwerke gültig.
\item Auf dem testents gibts faucts für entwickler, die einem ein wenig testnet cryptowährung überweisen nachdem man ein captcha gelöst hat. Problem öffters mal down.
\item Bei ethereum hatte ich das typische Java problem, dass die web3jlib eine neuere verion vewendete als der Jboss Wildfly. Daher musste ich den jboss hochziehen
\item Neuer wildfly benutzt port /127.0.0.1:9990 den auch ein nvidea network service verwendet. den musste ich dann erst abschiessen.
\end{enumerate}
\fi

\subsection{Grafische Benutzeroberfläche}


Die Webapplikation konsumiert den lokal laufenden Webservice. Hier dann mal schauen wie viel schneller es ist die buisness logic direkt aufzurufen und den Webservice zu umgehen. Die Resultate ebenfalls in die Masterarbeit aufnehmen.

    Hier das Model View Controller Pattern verwenden.
    Tapestry Wen Framework erklären
    GUI Schicht ist es egal ob sie Ethereum Daten oder DASH Daten bekommt. Stichwort Tapestry Service.
    Bild von MVC pattern und java Klassendiagramm
    Zeigen wie die Empfangsaddresse in den QR code kodiert ist. Hier könnte man auch das entsprechende BIP nennen.

\section{Evaluation}
\subsection{Prüfung der Anforderungen}

Dieser Abschnitt behandelt in wie weit das beschriebene Konzept die in Anschnitt \ref{anforderungen} aufgelisteten Anforderungen erfüllt. Die jeweilige Anforderung wird zunächst wiederholt und anschließend genauer untersucht.

\subsubsection{1) Transparente Einzahlungen}
\textit{Die Einzahlung jedes Endnutzers ist für jeden anderen Endnutzer nachprüfbar.}\\\\
Einzahlungen geschehen genau wie bei Bitcoin innerhalb von Transaktionen, die in die Blockchain geschrieben werden. Diese Anforderung ist also auch im Falle von Ethereum erfüllt.
\subsubsection{2) Gewinnerauswahl durch Zufallsfaktor}
\textit{Die Auswahl des Gewinners ist von einem zufälligen Faktor abhängig, auf den weder die Anwendung noch die Endnutzer einen Einfluss haben.}\\\\
Genau wie bei Bitcoin findet die Gewinnerauswahl basierend auf einem aus dem Proof-of-Work Blockhash statt. Die in Kapitel \ref{btc_evaluation} betrachtete Analyse gilt somit genau so für Ethereum außer, dass der Mining Reward 3 Ether beträgt und durchschnittlich alle 12 Sekunden ausgeschüttet wird. In Zukunft plant Ethereum von einem Proof-of-Work Algorithmus auf einen Proof-of-Stake Algorithmus umzusteigen. Proof-of-Stake und die daraus resultierenden Auswirkungen werden in Kapitel \ref{pos} betrachtet.
\subsubsection{3) Nachprüfbarkeit des Zufallsfaktor}
\textit{Jeder Endnutzer kann die Echtheit des zufälligen Faktors eigenständig nachprüfen.}\\\\
Der zur Gewinnerauswahl verwendete Blockhash ist zum Zeitpunkt der Auszahlung bereits in der öffentlichen Blockchain verankert und kann somit überprüft werden.
\subsubsection{4) Transparente Auszahlungen}
\textit{Die Auszahlung an den Gewinner muss transparent und somit für jeden Endnutzer nachprüfbar sein.}\\\\
Die Auszahlung wird durch die vom Nutzer initiierte \code{payout} Transaktion ausgelöst und vom Smart Contract vorgenommen. Das Verfahren der Auszahlung ist durch den unveränderlichen Smart Contract Code in Stein gemeißelt. Die Konsensregeln und die dahinter liegende Spieltheorie garantieren, dass dieser auch genau so ausgeführt wird. Obwohl die \code{payout} Transaktion nicht direkt Geld auf die Auszahlungsadresse des Gewinners überweist, kann der Endnutzer dennoch sicher sein, dass eine Auszahlung stattgefunden hat, wenn die \code{payout} Transaktion wie in Abbildung \ref{fig:contract_payout_txn} im Status \code{success} vorliegt.
\subsubsection{5) Fairheit des Spiels}
\textit{Jeder Endnutzer besitzt die gleiche Gewinnwahrscheinlichkeit und niemand wird benachteiligt.}\\\\
Damit keiner der Spieler einen Vorteil hat, muss jeder Topf-Platz die gleiche Gewinnwahrscheinlichkeit haben.
Dies ist gegeben, falls jeder Teilnehmer a) die gleiche Anzahl Gewinnzahlen zugeordnet bekommt und b) falls die möglichen Blockhash-Werte für die Gewinnerauswahl gleichverteilt sind.

a) Statt wie bei Bitcoin ausschließlich die letzte Ziffer des Blockhashs für die Gewinnerauswahl zu verwenden und als Konsequenz lediglich Töpfe der Größe 2, 5 und 10 anzubieten, wird bei Ethereum der gesamte Blockhash zur Gewinnerauswahl verwendet. Dies führt zu beliebig großen Töpfen, bei denen einige Teilnehmer genau eine Gewinnzahl mehr haben können. Da jeder Spieler in der Praxis mehrere Millionen von Gewinnzahlen hat, kann man diesen theoretischen Vorteil vernachlässigen.

b) Abschnitt \ref{eth_distribution} zeigt, dass die von Ethereum eingesetzte \code{Keccak-256} Hashfunktion gleichverteilte Werte liefert.


\subsection{Aufruf der Auszahlungstransaktion}
Wie bereits in Abschnitt \ref{eth_konzept} betrachtet, ist der Aufruf deiner Funktion zur Auszahlung unumgänglich. Da der Smart Contract dies nicht selber kann, muss der Aufruf entweder von außerhalb oder von einem Anderen Smart Contract kommen.

a) Aufruf von außerhalb:\\
Der Aufruf kann wie in der Implementierung vom Gewinner ausgeführt werden. In diesem Fall zahlt der Gewinner die Transaktionsgebühr und erhält den gesamten Topf-Betrag. Der Gewinner ist dafür zuständig die Funktion rechtzeitig aufzurufen, da der Gewinn sonst in den nächsten Topf übergeht. Eine andere Möglichkeit ist es, dass die Glücksspielanwendung den Smart Contract überwacht und die \textit{payout} Funktion rechtzeitig aufruft. In diesem Fall müsste die Transaktionsgebühr von der Glücksspielanwendung gezahlt werden oder Funktionalität in den Smart Contract eingebaut werden, die die Transaktionskosten vom Topf-Betrag abzieht und der Glücksspielanwendung zurückerstattet. Allerdings verlässt sich der Gewinner dann auf die Anwendung und geht dadurch ein Risiko ein.

b) Aufruf durch Smart Contract:\\
Man kann in der Theorie den Ansatz des Ethereum Alarm Clock Contracts \footnote{\url{https://etherscan.io/address/0x6c8f2a135f6ed072de4503bd7c4999a1a17f824b}} verwenden, um eine gewünschte Smart Contract Funktion zu einem späteren Zeitpunkt auszuführen. Man spezifiziert dazu welche Funktion man wann (in welchem Blockzeitraum) ausführen möchte und zahlt für die anfallenden Transaktionsgebühren im Voraus. Dies erlaubt, dass eine ganze Reihe von Funktionen sich bei dem Alarm Clock Contract registrieren. Wird nun der Alarm Clock Contract von einem durch einen privaten Schlüssel kontrollierten Account ausgelöst, werden alle registrierten Funktionen aufgerufen. Leider liefert diese Vorgehensweise keine  Garantie, da eine registrierte Funktion nur aufgerufen wird, falls der Alarm Clock Contract aufgerufen wird. Die Glücksspielanwendung müsste also einspringen, sobald niemand anderes bereit ist den Alarm Clock Contract anzustoßen. Es handelt sich also lediglich um eine Vorgehensweise um Transaktionsgebühren mit anderen Ethereum Nutzern zu teilen. Weitere Informationen zum Ethereum Alarm Clock Contract findet man unter \cite{eth_alarm_clock}.

\subsection{Verteilung der Hashfunktion Keccak-256}\label{eth_distribution}

Ethereum verwendet die kryptographische Hashfunktion Keccak-256.
Die folgende Monte-Carlo-Simulation zeigt, dass die Hashwerte der Hashfunktion Keccak-256 gleichverteilt sind.
\begin{verbatim}
h=Keccak-256 n=1000000
for i 1 -> n
    hash = h(i);
    result[uint(hash)%10]++
\end{verbatim}
\begin{minipage}{0.5\textwidth}
\begin{verbatim}
Ausgabe:
result[0] =  99227
result[1] = 100479
result[2] = 100163
result[3] =  99804
result[4] =  99945
result[5] = 100208
result[6] = 100403
result[7] = 100438
result[8] = 100035
result[9] =  99298
\end{verbatim}
\end{minipage}
\begin{minipage}{0.5\textwidth}
\includegraphics[width=\textwidth]{Figures/verteilung_keccak256}
\centering
\decoRule
\captionof{figure}{Verteilung der Keccak-256 Hashfunktion}
\label{fig:verteilung_keccak256}
\end{minipage}

\subsection{Sicherheit von Smart Contracts}
Bei Smart Contracts handelt es sich um öffentliche, für jeden ausführbare und unveränderliche Software. Beinhaltet diese einen Software Fehler, ist dieser ausnutzbar und kann nicht behoben werden. Smart Contracts verwalten in der Regel Geld oder Token, die einen finanziellen Wert repräsentieren. Bei der Entwicklung eines Smart Contracts ist somit oberste Vorsicht geboten. \todo{Hier zuerst: Durch einen kritischen Fehler im the DAO smart Contratct wurden Millionen verloren... dann erst details was the DAO ist.vllt als fussnote} Das Beispiel von the DAO zeigt, zu welchen katastrophalen Folgen Sicherheitslücken in Smart Contracts führen können.
Bei the DAO handelt es sich um einen von Christoph Jentzsch programmierten und am 20ten April 2016 auf der Ethereum Blockchain veröffentlicht Smart Contract\footnote{\url{https://etherscan.io/address/0xbb9bc244d798123fde783fcc1c72d3bb8c189413\#code}}.
The DAO ist ein Kapitalfond, der es sich zur Aufgabe gemacht hat in Blockchain Technologie zu investieren. Bei dem initialen 28zig tägigen Crowdsale wurden mehr als 150 Millionen Dollar von über 11 Tausend Investoren eingesammelt. Investoren haben Kapital in Form von Ether eingezahlt und als Gegenleistung eine entsprechende Anzahl Token als eine Art Stimmrecht erhalten. Investitionsentscheidungen dieser \textbf{d}ezentralen \textbf{a}utonomen \textbf{O}rganisation werden mithilfe des Smart Contracts durch einen dezentral erarbeiteten Konsens getroffen. Durch das Ausnutzen eines nicht trivialen Fehlers im Smart Contract Code schaffte es ein Hacker einen großen Teil des Kapitals an eine von ihm kontrollierte Adresse auszuzahlen. Eine genaue Beschreibung des Angriffes findet man unter \cite{eth_dao_hack}.

Die Solidity Dokumentation \footnote{\url{https://solidity.readthedocs.io/en/develop/security-considerations.html}} listet eine Reihe von Beispielen, die die Sicherheit von Smart Contracts betreffen. Entwickler sollten sich dieser bewusst sein, bevor sie einen Smart Contract veröffentlichen der Geld verwaltet.
\todo{Sicherheitsaspekt auf eigenen Smart Contract beziehen.}

\if
Hier dann noch darauf eingehen, dass man auf keinen Fall bock.timestamp verwenden sollte, da Miner auf diesen einen direkten Zugriff haben können.

https://ethereum.stackexchange.com/questions/19341/address-send-vs-address-transfer-best-practice-usage

Hier dann nur darauf eingehen, dass falls man die Auszahlung mittels der unsicheren Methode macht ein BUG besteht. 

Anscheinend ist bei address.transfer aber sicher, dass kein contract code ausgeführt wird. Es gibt allerdings 3 Möglichkeiten zu senden.
Eine ist unsicher.

\begin{lstlisting}
function payout() public{
    assert(potClosed);
    assert(block.number>payoutBlockNumber);
    potClosed = false; //fixes bug
    payoutBlockHash = block.blockhash(payoutBlockNumber); 
    if(payoutBlockHash == 0){
        nbrOfMissedPayouts++;
    }else{
        winner = uint(payoutBlockHash) % NBR_OF_SLOTS;
        address winnerAddress = payoutAddresses[winner];
        uint amount= EXPECTED_POT_AMOUNT*NBR_OF_SLOTS;
        amount += EXPECTED_POT_AMOUNT*NBR_OF_SLOTS*nbrOfMissedPayouts;
        winnerAddress.transfer(amount); // send pot amount to winner
        nbrOfMissedPayouts = 0;
    }
    nbrOfParticipants=0;
}
\end{lstlisting}


\fi


\chapter{Sonstige Blockchain-Technologie} % Main chapter title
\section{Directed acyclic graph}
Hier dann darauf eingehen, dass solch ein Ansatz keinen Sinn macht.
\section{Konsensalgorithmus: Proof of stake }\label{pos}
Hier kann man auch noch erwähnen, dass Proof of stake und solche slotbasierten Ansätze nicht geeignet sind da der Slotleader direkten Einfluss nehmen kann.
\section{Payment Channels und Lightning Network}
Hier könnte man darauf eingehen, dass off-chain Transaktionen nicht einsetzbar sind, da die restlichen Teilnehmer somit nicht die Einzahlung überprüfen können.

\chapter{Vorhandene Glücksspielseiten}
\todo{Ich habe in der Einleitung versprochen, dass hier noch das betrachtete Verfahren bestehender Glücksspielseiten betrachtet wird.}
\section{Bitcoin}

\section{Ethereum}
\chapter{Ausblick} % Main chapter title

\label{ausblick} % For referencing the chapter elsewhere, use \ref{ausblick} 

\iffalse
\section{Integration in Wallet}

Die eigentliche Einzahlung in den Pot muss von solch einem Client gemacht werden. Zeigen wie die Wallet-App den Webservice konsumiert. Zeigen wie die Wallet-App eine Einzahlung macht.

\section{Andere Anwendungsgebiete}

\fi 

\chapter{Fazit} % Main chapter title
Der Einsatz von Blockchain-Technologie ermöglicht
den Austausch von digitalen finanziellen Werten zwischen sich gegenseitig misstrauenden Parteien, ohne dabei auf eine Trusted Third Party angewiesen zu sein.
Das vorher benötigte Vertrauen wird nun in ein öffentliches, transparentes System verlagert, dessen inhärente Spieltheorie den Teilnehmern finanzielle Anreize liefert, sich korrekt zu verhalten. 

Diese Arbeit hat zunächst gezeigt, wie man solche Systeme in eine eigene Anwendung integrieren kann und auf welche Besonderheiten dabei zu achten ist. 
Außerdem wurde aufgezeigt, dass man Systeme, die auf einem Proof-of-Work Konsensalgorithmus basieren, in einem gewissen Rahmen als eine verlässliche Zufallsquelle nutzen kann.
Die im ersten Ansatz entwickelte, auf der Bitcoin Blockchain aufbauende Glücksspielanwendung erlaubt es dem Nutzer die zufällige Gewinnerauswahl nachzuprüfen. Die Anwendung kann den Nutzer in dieser Hinsicht nicht benachteiligen oder betrügen. Lediglich die von der Anwendung vorzunehmende Auszahlungstransaktion an den Gewinner bietet eine gewisse Angriffsfläche. Der Endnutzer muss der Anwendung vertrauen, diese korrekt durchzuführen. Eine korrumpierte, sich falsch verhaltende Anwendung fällt dem Endnutzer allerdings auf. Ein solcher Service ist nur unter der Bedingung nutzbar, dass der Endnutzer den Betreiber des Services kennt und somit juristisch haftbar machen kann. 

Diese Problematik wurde mithilfe sogenannter Smart Contracts der Ethereum Blockchain gelöst.
Diese erlauben es dem Nutzer vollständig auf Vertrauen verzichten zu können, da die Geschäftslogik der Glücksspielanwendung in der Blockchain verankert ist und von allen Teilnehmern des Netzwerkes ausgeführt wird. Der komplette Verzicht auf Vertrauen resultiert allerdings in einer schlechteren Usability. Der Nutzer muss verstehen, welche Smart Contract Funktion er zu welchem Zeitpunkt aufrufen muss. Verpasst er den korrekten Zeitpunkt, verliert er seinen Gewinn. Dies ist bei Bitcoin nicht der Fall. Dort muss er lediglich einen QR-Code scannen und die Zahlung autorisieren.\\\\


Das Ziel von Blockchains ist es durch transparente Systeme, die Interaktionen zwischen sich misstrauenden Parteien zu ermöglichen ohne, dass dabei das Vertrauen in eine Drittpartei erforderlich ist. Das Beispiel der Glücksspielanwendung ist in so weit gelungen, da keine zusätzlichen Daten aus der echten Welt dafür benötigt werden. Im Falle von Ethereum findet die gesamte Interaktion innerhalb des Ethereum Systems statt.

Andere Anwendungsfälle wie Supply Chain, ... sind schwerer zu realisieren, da man auf Daten angewiesen ist, die aus der echten Welt in die Blockchain geschrieben werden muss. An der Schnittstelle zwischen der auf Kryptographie basierenden Blockchain Welt und der echten Welt ist es unmöglich vollständig auf Vertrauen zu verzichten.\\\\

TODO:\\
Ansprechen, dass meine Projektidee gezwungenermaßen on-chain-Transaktionen braucht, da sie sonst nicht von allen Nutzern nachvollzogen werden können. Da Bitcoin mit sehr großer Wahrscheinlichkeit weiterlebt und nicht in nächster Zukunft nicht ''stirbt'' werden bei steigendem Bedarf on-chain-Transaktionen stetig teurer. Daraus folgt dann, dass meine Idee keinen Sinn mehr für kleine Beträge macht. Hier dann nochmal darauf eingehen, dass das Lightning Network verspricht einen Großteil der Transaktionen off-cain zu bringen.



%----------------------------------------------------------------------------------------
%	THESIS CONTENT - APPENDICES
%----------------------------------------------------------------------------------------

\appendix % Cue to tell LaTeX that the following "chapters" are Appendices

% Include the appendices of the thesis as separate files from the Appendices folder
% Uncomment the lines as you write the Appendices

%\include{Appendices/AppendixA}
%\include{Appendices/AppendixB}
%\include{Appendices/AppendixC}

%----------------------------------------------------------------------------------------
%	BIBLIOGRAPHY
%----------------------------------------------------------------------------------------

\printbibliography[heading=bibintoc,title={Quellenverzeichnis}]


%----------------------------------------------------------------------------------------
%	LIST OF FIGURES/TABLES
%----------------------------------------------------------------------------------------


\listoffigures % Prints the list of figures

%\listoftables % Prints the list of tables


\end{document}  
