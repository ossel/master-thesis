\chapter{Fazit} % Main chapter title
Diese Arbeit hat am Beispiel einer Glücksspielanwendung gezeigt, dass der Einsatz von Blockchain-Technologie es anonymen, sich gegenseitig nicht vertrauenden Parteien ermöglichen kann, direkt miteinander zu interagieren, ohne dass dabei eine Trusted Third Party als Mittelsmann verwendet werden muss. Das vorher benötigte Vertrauen wurde in ein öffentliches, transparentes System verlagert, dessen inhärente Spieltheorie den Teilnehmern finanzielle Anreize liefert sich korrekt zu verhalten.

Diese Arbeit hat zunächst gezeigt, wie man solche Systeme in eine eigene Anwendung integrieren kann und auf welche Besonderheiten dabei zu achten ist. 
Außerdem wurde aufgezeigt, dass man Systeme, die auf einem Proof-of-Work Konsensalgorithmus basieren, in einem gewissen Rahmen als eine verlässliche Zufallsquelle nutzen kann. Die im ersten Ansatz entwickelte, auf der Bitcoin Blockchain aufbauende Glücksspielanwendung erlaubt es dem Nutzer die zufällige Gewinnerauswahl nachzuprüfen. Die Anwendung kann den Nutzer in dieser Hinsicht nicht benachteiligen oder betrügen. Lediglich die von der Anwendung vorzunehmende Auszahlungstransaktion an den Gewinner bietet eine gewisse Angriffsfläche. Der Endnutzer muss der Anwendung vertrauen, dass diese die Auszahlung korrekt ausführt. Eine korrumpierte, sich falsch verhaltende Anwendung fällt dem Endnutzer allerdings auf. 
%Ein solcher Service ist nur unter der Bedingung nutzbar, dass der Endnutzer den Betreiber des Services kennt und somit juristisch haftbar machen kann.
Diese Problematik wurde mithilfe sogenannter Smart Contracts der Ethereum Blockchain gelöst. Diese erlauben es dem Nutzer vollständig auf Vertrauen verzichten zu können, da die Geschäftslogik der Glücksspielanwendung in der Blockchain verankert ist und von allen Teilnehmern des Netzwerks ausgeführt wird. Durch den Einsatz der Ethereum Blockchain wurde das Ziel, vollständig auf eine Trusted Third Party zu verzichten, erreicht.\\

\noindent Das Beispiel der Glücksspielanwendung auf der Ethereum Plattform ist in soweit gelungen, da keine zusätzlichen Daten aus der echten Welt für diesen Anwendungsfall benötigt wurden. Die gesamte Interaktion findet innerhalb des Ethereum Netzwerks statt und wird durch den bewährten Einsatz von Kryptographie abgesichert. Andere Anwendungsfälle für Blockchain-Technologie wie beispielsweise Supply-Chain-Management oder Logistik und Transport sind schwieriger zu realisieren, da man auf Daten aus der echten Welt angewiesen ist. An der Schnittstelle, die diese Daten in die Blockchain schreibt und somit dort verfügbar macht, ist es unmöglich vollständig auf Vertrauen zu verzichten.\\

\noindent Ein weiterer offene Frage ist, ob es Blockchains in Zukunft schaffen werden, ausreichend zu skalieren. Sogenannte Second Layer Solutions bieten einen Ansatz valide Transaktionen außerhalb der Blockchain zu akzeptieren und nur den resultierenden Zustand in die Blockchain zu schreiben. Ob dies ausreicht, die eigentliche Blockchain genügend zu entlasten, damit On-Chain Transaktionen langfristig bezahlbar bleiben, kann nur die Zeit zeigen.

%Der Einsatz von Blockchain-Technologie ermöglicht den Austausch von digitalen finanziellen Werten zwischen sich gegenseitig nicht vertrauenden Parteien, ohne dabei auf eine Trusted Third Party angewiesen zu sein.