\chapter{Fazit} % Main chapter title
Der Einsatz von Blockchain-Technologie ermöglicht
den Austausch von digitalen finanziellen Werten zwischen sich gegenseitig misstrauenden Parteien, ohne dabei auf eine Trusted Third Party angewiesen zu sein.
Das vorher benötigte Vertrauen wird nun in ein öffentliches, transparentes System verlagert, dessen inhärente Spieltheorie den Teilnehmern finanzielle Anreize liefert, sich korrekt zu verhalten. 

Diese Arbeit hat zunächst gezeigt, wie man solche Systeme in eine eigene Anwendung integrieren kann und auf welche Besonderheiten dabei zu achten ist. 
Außerdem wurde aufgezeigt, dass man Systeme, die auf einem Proof-of-Work Konsensalgorithmus basieren, in einem gewissen Rahmen als eine verlässliche Zufallsquelle nutzen kann.
Die im ersten Ansatz entwickelte, auf der Bitcoin Blockchain aufbauende Glücksspielanwendung erlaubt es dem Nutzer die zufällige Gewinnerauswahl nachzuprüfen. Die Anwendung kann den Nutzer in dieser Hinsicht nicht benachteiligen oder betrügen. Lediglich die von der Anwendung vorzunehmende Auszahlungstransaktion an den Gewinner bietet eine gewisse Angriffsfläche. Der Endnutzer muss der Anwendung vertrauen, diese korrekt durchzuführen. Eine korrumpierte, sich falsch verhaltende Anwendung fällt dem Endnutzer allerdings auf. Ein solcher Service ist nur unter der Bedingung nutzbar, dass der Endnutzer den Betreiber des Services kennt und somit juristisch haftbar machen kann. 

Diese Problematik wurde mithilfe sogenannter Smart Contracts der Ethereum Blockchain gelöst.
Diese erlauben es dem Nutzer vollständig auf Vertrauen verzichten zu können, da die Geschäftslogik der Glücksspielanwendung in der Blockchain verankert ist und von allen Teilnehmern des Netzwerkes ausgeführt wird. Der komplette Verzicht auf Vertrauen resultiert allerdings in einer schlechteren Usability. Der Nutzer muss verstehen, welche Smart Contract Funktion er zu welchem Zeitpunkt aufrufen muss. Verpasst er den korrekten Zeitpunkt, verliert er seinen Gewinn. Dies ist bei Bitcoin nicht der Fall. Dort muss er lediglich einen QR-Code scannen und die Zahlung autorisieren.\\\\


Das Ziel von Blockchains ist es durch transparente Systeme, die Interaktionen zwischen sich misstrauenden Parteien zu ermöglichen ohne, dass dabei das Vertrauen in eine Drittpartei erforderlich ist. Das Beispiel der Glücksspielanwendung ist in so weit gelungen, da keine zusätzlichen Daten aus der echten Welt dafür benötigt werden. Im Falle von Ethereum findet die gesamte Interaktion innerhalb des Ethereum Systems statt.

Andere Anwendungsfälle wie Supply Chain, ... sind schwerer zu realisieren, da man auf Daten angewiesen ist, die aus der echten Welt in die Blockchain geschrieben werden muss. An der Schnittstelle zwischen der auf Kryptographie basierenden Blockchain Welt und der echten Welt ist es unmöglich vollständig auf Vertrauen zu verzichten.\\\\

TODO:\\
Ansprechen, dass meine Projektidee gezwungenermaßen on-chain-Transaktionen braucht, da sie sonst nicht von allen Nutzern nachvollzogen werden können. Da Bitcoin mit sehr großer Wahrscheinlichkeit weiterlebt und nicht in nächster Zukunft nicht ''stirbt'' werden bei steigendem Bedarf on-chain-Transaktionen stetig teurer. Daraus folgt dann, dass meine Idee keinen Sinn mehr für kleine Beträge macht. Hier dann nochmal darauf eingehen, dass das Lightning Network verspricht einen Großteil der Transaktionen off-cain zu bringen.