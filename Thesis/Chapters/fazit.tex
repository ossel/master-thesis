\chapter{Fazit} % Main chapter title
Die das erste man in Bitcoin verwendete 
Blockchain-Technologie
ermöglicht es in Kombination mit einem
durch den Einsatz eines Proof of Work Algorithmus 
vollständig auf
Trusted Third Parties zu verzichten.
indem man Vertrauen 
in ein öffentliches transparentes System verlagert,
das den Teilnehmern des Systems Anreize liefert sich korrekt zu verhalten.\\\\

Die auf der Bitcoin Blockchain aufbauende Glücksspielanwendung besitzt eine gewisse Angriffsfläche. Dies führt dazu, dass der Benutzer der Anwendung zu einem gewissen Grad vertrauen muss. Eine korrumpierte, sich falsch verhaltende Anwendung fällt dem Endnutzer allerdings auf. Ein solcher Service ist nur unter der Bedingung nutzbar, dass der Endnutzer den Betreiber des Services kennt und somit juristisch haftbar machen kann. 
Mit Ethereum kann dagegen komplett auf Vertrauen verzichten werden, da die Geschäftslogik der Glücksspielanwendung in der Blockchain verankert und von allen Teilnehmern des Netzwerkes ausgeführt wird. Der komplette Verzicht auf Vertrauen resultiert allerdings in einer schlechteren Usability. Der Nutzer muss verstehen, welche Smart Contract Methode er zu welchem Zeitpunkt aufrufen muss. Verpasst er den korrekten Zeitpunkt, verliert er seinen Gewinn. Dies ist bei Bitcoin nicht der Fall. Dort muss er lediglich einen QR-Code scannen und die Zahlung autorisieren.\\\\\\


Ansprechen, dass meine Projektidee gezwungenermaßen on-chain-Transaktionen braucht, da sie sonst nicht von allen Nutzern nachvollzogen werden können. Da Bitcoin mit sehr großer Wahrscheinlichkeit weiterlebt und nicht in nächster Zukunft nicht ''stirbt'' werden bei steigendem Bedarf on-chain-Transaktionen stetig teurer. Daraus folgt dann, dass meine Idee keinen Sinn mehr für kleine Beträge macht. Hier dann nochmal darauf eingehen, dass das Lightning Network verspricht einen Großteil der Transaktionen off-cain zu bringen.
\todo{Letzten Absatz richtig machen}