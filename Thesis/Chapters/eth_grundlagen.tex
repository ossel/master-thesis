\chapter{Zweiter Ansatz: Ethereum} % Main chapter title

\label{eth} % For referencing the chapter elsewhere, use \ref{eth} 

\section{Grundlagen}
Ethereum ist genau wie Bitcoin ein Peer-to-Peer Netzwerk Protokoll, dass auf einer öffentlichen Blockchain basiert und einen Systemzustand verwaltet. Es handelt sich im Gegensatz zu Bitcoin um eine generalisierte Blockchain die nicht nur Finanztransaktionen speichern kann, sondern sogenannte Smart Contracts. Ethereum sieht sich als eine Open Source Plattform, die es ermöglicht dezentrale Applikationen zu entwickeln und bereitzustellen.

\subsection{Smart Contracts}
Ethereum ermöglicht es Geschäftsprozesse in Form von Smart Contracts zu programmieren und von einem globalen dezentralen Netzwerk ausführen zu lassen. Ein Contract ist ein Programm bestehend aus einer Reihe von Anweisungen, die ausgeführt werden, sobald das Programm eine Mitteilung in Form einer Transaktion erhält. Contracts haben die Möglichkeit Daten aus der Blockchain auszulesen und auf ihr Daten zu speichern. Smart Contracts können sowohl von Menschen als auch von anderen Contracts ausgelöst werden. Contracts können daher mit anderen Contracts über die vom Programmierer festgelegte Schnittstelle interagieren.
\subsection{Ethereum Accounts}
Um Ethereum nutzen zu können braucht man einen Account. Ethereum Accounts haben:\\
\begin{itemize}
\item Eine 20 Byte lange Adresse,
\item Einen Kontostand in Ether,
\item Contract Code der durch den Account verwaltet wird,
\item Speicherplatz,
\end{itemize} In Ethereum gibt es 2 Arten von Accounts. Die erste Art ist eine Adresse die durch den Besitzer des privaten Schlüssels kontrolliert wird. Diese Art von Account ist vergleichbar mit einer Bitcoin Adresse, die man durch eine Bitcoin-Wallet kontrolliert. Die zweite Art ist ein Contract Account, der durch den Contract Code kontrolliert wird. Der Contract Code verwaltet eigenständig den Kontostand des Accounts und verhält sich genauso wie der Programmierer es festgelegt hat.
\subsection{Transaktionen}
Interaktionen mit dem Ethereum Netzwerk finden in Form von sogenannten Transaktionen statt. Transaktionen beinhalten:
\begin{itemize}
\item eine Empfangsadresse.
\item eine Anzahl an Ether die versandt wird.
\item Daten (Bytes), die vom Contract ausgelesen werden können.
\end{itemize}

Es gibt zwei verschiedene Arten von Transaktion. Sie unterscheiden sich durch die in der Transaktion angegebene Empfangsadresse. 
Wird die Empfangsadresse durch einen privaten Schlüssel kontrolliert, handelt es sich lediglich um eine Finanztransaktion, die Ether vom Sender zum Empfänger Account transferiert. Anderenfalls handelt es sich um eine Transaktion, die den Contract Code der Empfangsadresse ausführt. Welche Funktion ausgeführt werden soll, wird vom Sender im Datenfeld der Transaktion kodiert. Jeder Netzwerkknoten arbeitet jede einzelne Transaktion auf der Blockchain ab und speichert den gesamten resultierenden Status. 
\subsection{Ethereum Virtual Machine}
In ihr wird der Code der Contracts ausgeführt. Sie arbeitet die Anweisungen des Contracts der Reihe nach ab und bricht ab, falls in der Transaktion nicht genügend Ether für die Ausführung des Codes bezahlt wurden.
\subsection{Ether}. 
Ist die Kryptowährung des Ethereum Netzwerks. Sie wird benutzt um für Transaktionsgebühren und die Ausführung von Contract Code zu bezahlen. Ether entsteht ähnlich wie bei Bitcoin durch Mining.
\subsection{Ethereum Client}. 
Einen Ethereum Client wird benötigt. um am Netzwerk teilnehmen zu können. Vor der Verwendung muss dieser sich erst mit dem Netzwerk synchronisieren. Bei der Synchronisation werden die neuen Transaktionen heruntergeladen und überprüft. Weit verbreitete Ethereum Clients sind Ethereum Wallet und Mist.
