\chapter{Einleitung} % Main chapter title
\label{Chapter1}
%----------------------------------------------------------------------------------------
\newcommand{\keyword}[1]{\textbf{#1}}
\newcommand{\tabhead}[1]{\textbf{#1}}
\newcommand{\code}[1]{\texttt{#1}}
\newcommand{\file}[1]{\texttt{\bfseries#1}}
\newcommand{\option}[1]{\texttt{\itshape#1}}
%----------------------------------------------------------------------------------------
Die Erfindung der Kryptowährung Bitcoin und deren inhärente Blockchain-Technologie hat in den letzten Jahren einen regelrechten Hype ausgelöst. Begriffe wie Blockchain, Distributed Ledger und Smart Contract sind das bestimmende Thema schlechthin. 
Blockchain-Technologie verspricht, durch den Einsatz von Dezentralität, Unveränderlichkeit und Transparenz, die Möglichkeit zwischen anonymen, sich gegenseitig nicht trauenden, Parteien digitale Werte auszutauschen.

\section{Motivation}
Das Internet vernetzt weltweit mehrere Millionen Menschen. Es ermöglicht diesen anonym zu kommunizieren und miteinander zu interagieren. Aufgrund der Anonymität und des damit einhergehenden Betrugspotential, benötigen die meisten Anwendungsfälle eine sogenannte Trusted Third Party als Mittelsmann. Das fehlende Vertrauen zwischen den anonym agierenden Parteien wird durch diese dritte, nicht anonym auftretende, Partei kompensiert. Diese nicht anonyme Partei muss sich den länderspezifischen Regeln und Gesetzen unterwerfen und kann mit ihrem Service daher nur in einem gewissen, vorgegebenen Rahmen betreiben. Da das Betreiben eines solchen Services mit gewissen Kosten verbunden ist, werden diese meist auf die Nutzer abgewälzt. 

Online Casinos sind ein Beispiel für eine solche Trusted Third Party. Diese bieten eine Plattform, die Spieler vernetzt und vor gegenseitigem Betrug schützt. Außerdem verwalten Online Casinos das Geld ihrer Spieler. Die Spieler vertrauen den Casinos diese Aufgabe an, da es sich um registrierte Firmen handelt, die juristisch haftbar gemacht werden können. Ein weiterer Aspekt bei dem Vertrauen eine Rolle spielt sind die angebotenen Spiele selbst. Egal ob Black Jack, Poker oder Roulette, die meisten solcher Spiele erfordern die Generierung von Zufallszahlen. Beispielsweise wenn Spielkarten verteilt werden. Die Spieler haben dabei keine Möglichkeit nachzuprüfen, ob der Algorithmus, der ihnen die Karten zuteilt, auch wirklich fair ist. Die Spieler müssen dem Casino somit vertrauen, dass dieses sie nicht benachteiligt. 

%Der Endnutzer wird somit durch Transparenz und nicht durch staatliche Regulierungen geschützt. 
\section{Ziel}
Ziel dieser Masterarbeit ist es den Einsatz von Blockchain-Technologie an der beispielhaften Realisierung einer Glücksspielanwendung zu demonstrieren. Der Einsatz einer Blockchain soll dabei das Vertrauen, das der Endnutzer der Anwendung entgegenbringen muss, auf ein Minimum reduzieren. Falls möglich soll vollständig auf den Einsatz einer Trusted Third Party verzichtet werden.


%Das Ziel von Blockchains ist es durch transparente Systeme, die Interaktionen zwischen sich misstrauenden Parteien zu ermöglichen ohne, dass dabei das Vertrauen in eine Drittpartei erforderlich ist.

% Hier ein paar unwissenschaftliche Gründe warum 
%Als Spieler muss man sich erst registrieren und seine Bankdaten angeben. Durch diese Registrierung gibt der Spieler seine Anonymität auf. Außerdem hat der Staat eine zentrale Stelle an die er sich wenden kann um Steuereinnahmen durch die Besteuerung von Glücksspiel zu generieren. Ein weiterer Punkt der bei derzeitig etablierten Glücksspielplattformen verbesserungswürdig ist, sind Ein- und Auszahlungen. Einzahlungen und Auszahlungen  per Banküberweisungen dauern mindestens einen Tag. Dadurch muss der Glücksspieler eine lange Zeit warten bevor er den Service nutzen kann. Auch potentielle Gewinne kann er erst nach solch einer langen Zeitdauer weiterverwenden. Heutzutage gibt es zwar Dienste wie PayPal, die schnelle Zahlungen realisieren, allerdings kostet die Integration solch eines Zahlungsmittels recht hohe Gebühren für die Glücksspielplattform. Cryptowährungen bieten in dieser Hinsicht Abhilfe, da Transaktionen innerhalb wenigen Minuten bestätigt werden. Der Glücksspieler kann beispielsweise in einem Crypto-Casino spielen und sich anschließend von seinen Gewinnen eine Pizza bei Lieferando kaufen. :) (Hier nochmal drüber nachdenken ob man die Argumente nicht wissenschaftlicher verpacken könnte)

\section{Projektidee}
Die in dieser Arbeit betrachtete Glücksspielanwendung soll ein Spiel anbieten, bei dem Teilnehmer in einen Geldtopf einzahlen und auf ein zufälliges Event wetten. Jeder der Teilnehmer soll dabei die gleichen Gewinnchancen haben. Sobald alle Teilnehmer eingezahlt haben, wird einer der Teilnehmer zufällig ausgewählt und gewinnt den gesamten Geldtopf. Der Gewinner bekommt somit seinen eigenen Einsatz als auch den Einsatz aller Mitspieler ausgezahlt. Die restlichen Teilnehmer verlieren und gehen leer aus.\\\\

\noindent Die erstmalig in Bitcoin verwendete Blockchain Technologie ist für die Entwicklung einer solchen Anwendung bestens geeignet, da sie transparente, pseudonyme Zahlungen ermöglicht. Außerdem lässt sich der für die Gewinnerauswahl benötigte Zufall durch ein in der Zukunft liegenden Zustand der Blockchain abbilden. 
Der Zufallsfaktor kommt somit direkt von der Blockchain und ist für alle Teilnehmer nachvollziehbar.

%Was genau das eigentliche Spiel für den Nutzer der Glücksspielanwendung ist, ist für diese Masterarbeit zweitrangig. Ziel dieser Masterarbeit ist es zu erforschen, welche Möglichkeiten die erstmals in Bitcoin verwendete Blockchain Technologie bietet und in wie weit durch sie das Vertrauen des Endnutzers in die Anwendung reduziert werden kann.


\section{Anforderungen}\label{anforderungen}
Die zu realisierende Glücksspielanwendung muss den folgenden Anforderungen gerecht werden.
\subsubsection{1) Transparente Einzahlungen}
Die Einzahlung jedes Endnutzers ist für jeden anderen Endnutzer nachprüfbar.
\subsubsection{2) Gewinnerauswahl durch Zufallsfaktor}
Die Auswahl des Gewinners ist von einem zufälligen Faktor abhängig, auf den weder die Anwendung noch die Endnutzer einen Einfluss haben.
\subsubsection{3) Nachprüfbarkeit des Zufallsfaktor}
Jeder Endnutzer kann die Echtheit des zufälligen Faktors eigenständig nachprüfen.
\subsubsection{4) Transparente Auszahlungen}
Die Auszahlung an den Gewinner muss transparent und somit für jeden Endnutzer nachprüfbar sein.
\subsubsection{5) Fairheit des Spiels}
Jeder Endnutzer besitzt die gleiche Gewinnwahrscheinlichkeit und niemand wird benachteiligt.

\section{Vorhandenes}

\subsection{Cyberdice Protokoll}

Einen ersten Ansatz wie man im Internet Glücksspiel ohne eine vertrauenswürdige Drittpartei betreiben kann, liefert \cite{cyberdice_paper}. Es stellt ein Kommunikationsprotokoll vor, das mit Hilfe kryptographischer Methoden sicherstellt, dass weder die Teilnehmer noch Außenstehende betrügen können. Das zum Glücksspiel verwendete Protokoll funktioniert aber nur unter der Annahme, dass es eine zentrale Institution (Bank) gibt, bei der die Teilnehmer Geld einzahlen und im Falle eines Gewinns gegen die Vorlage eines Beweises Geld ausgezahlt bekommen. Durch die Erfindung dezentraler Kryptowährungen, die auf einer für jeden einsehbaren Blockchain basieren, fällt diese vorher noch benötigte zentrale Institution weg.

%Durch die Erfindung dezentraler, Ländergrenzen überschreitender Krypto-Währungssysteme ist es möglich geworden den Einfluss des Staates zu umgehen und den Endnutzer durch die Transparenz der Glücksspielanwendung zu schützen.

\subsection{Glücksspielseiten}
Es gibt bereits Services die dezentrales, transparentes Glücksspiel mit Hilfe von Kryptowährungen umsetzen.

Die Internetseite \cite{crypto_games} bietet diverse Spiele an, bei denen die Nutzer die Möglichkeit haben mit Kryptowährungen zu bezahlen.

Die Internetseite \cite{vdice} bietet ein Würfelspiel an, das durch einen Smart Contract auf der Ethereum Plattform umgesetzt ist.

Eine genaue Beschreibung der verwendeten Verfahren zur Gewinnerauswahl befindet sich im Anhang dieser Ausarbeitung.

\section{Aufbau dieser Arbeit}
In dieser Arbeit wird zunächst Bitcoin als Beispiel einer auf Blockchain-Technologie basierenden Kryptowährung betrachtet. Anschließend wird der Einsatz von Smart Contracts mit Ethereum präsentiert. In beiden Fällen werden zunächst die relevanten Grundlagen geklärt und ein Konzept vorgestellt. Anschließend wird dieses als Glücksspielanwendung mithilfe der jeweiligen Technologie umgesetzt. Die resultierenden Glücksspielanwendungen werden letztendlich unter Zuhilfenahme der aufgestellten Anforderungen evaluiert. Nach diesem Hauptteil wird abschließend weitere Blockchain-Technologien und deren Anwendbarkeit für die Projektidee betrachtet.