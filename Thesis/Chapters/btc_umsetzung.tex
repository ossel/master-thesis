\section{Umsetzung}

\subsection{Interaktion mit dem Bitcoin Netzwerk}

Möchte man mit dem Bitcoin Netzwerk kommunizieren benötigt man einen Client der das Bitcoin Protokoll implementiert. Dieser kommuniziert dann mit dem Peer-to-Peer Netzwerk und wird dadurch zu einem Knoten (''Node'') des Peer-to-Peer Netzwerks. Man unterscheidet zwischen sogenannten ''Full Nodes'' und ''Light Nodes''.
\subsubsection{Full Node}
Knoten die eigenständig alle Transaktionen und Blöcke auf Gültigkeit mit Hilfe der Konsensregeln prüfen nennt man ''Full Node''. Diese Knoten speichern die gesamte Blockchain und bilden das Rückgrat des Netzwerkes.
Full Nodes stellen in der Regel eine RPC Schnittstelle zur Verfügung. Diese Schnittstelle bietet die Möglichkeit von einer beliebigen Programmiersprache aus mit dem Full node zu interagieren.
Abbildung \ref{fig:btc_core_full_node_architecture}\cite{mastering_bitcoin} zeigt, dass man über die RPC Schnittstelle auf die gespeicherten Daten des Nodes (Blöcke, Blockheader und Adressen) als auch auf die Wallet zugreifen kann.

\begin{figure}[H]
\centering
\includegraphics[width=1\linewidth]{Figures/btc_core_full_node_architecture}
\decoRule
\caption{Bitcoin Core: Full Node Aufbau}
\label{fig:btc_core_full_node_architecture}
\end{figure}

Abbildung \ref{fig:btc_core_full_node_architecture} zeigt außerdem: 
\begin{itemize}
\item Peer Discovery, Peer Datenbank und Connection Manager: Diese kümmern sich um die Kommunikation mit dem Peer-to-Peer Netzwerk.
\item Mempool: Im sogenannten Mempool werden empfangene, unbestätigte Transaktionen im Speicher gehalten.
\item Validation Engine: Diese validiert, ob die empfangenen Blöcke und deren Transaktionen die Konsensregeln einhalten. Falls ja werden die somit bestätigten Transaktionen aus dem Mempool gelöscht und der Block wird an die StorageEngine zur Abspeicherung in der Blockchain Datenbank weitergereicht.
\item Miner: Die Bitcoin Full Node Software enthält einen CPU Miner mit der man mithilfe des Proof-of-Work Algorithmus nach neuen Blöcken suchen kann. Bitcoin Mining ist heutzutage nur noch mit sogenannten ASICs profitabel. ASIC steht für \textbf{A}pplication-\textbf{S}pecific \textbf{I}ntegrated \textbf{C}ircuit. Es handelt sich um Hardware, die auf die Berechnung der SHA256 Hashfunktion spezialisiert ist.
\end{itemize}


\subsubsection{Light Node}
''Light Nodes'' speichern nicht die gesamte Blockchain, sonder in der Regel nur die Blockheader der Blöcke der Blockchain. Beim Mining gehen nur die Daten des Blockheaders in den Blockhash ein. Der Node empfängt Blockheader, prüft ihre Gültigkeit und fügt sie gegebenenfalls in die Headerkette ein. Der Node kann somit eigenständig, d.h. ohne seinen Nachbarn vertrauen zu müssen, die längste Proof-of-Work Kette bilden. Da diese Kette nur aus Headern besteht und keine Transaktionen enthält, kann der Light Node empfangene Transaktionen nicht eigenständig auf ihre Gültigkeit prüfen. Light Nodes verwenden das in \cite{bitcoin_white_paper} beschriebene ''Simplified Payment Verification'' Verfahren zur Prüfung von Transaktionen. ''Light Nodes'' werden daher oft auch ''SPV Client'' genannt. Ein ''SPV Client'' prüft die Gültigkeit einer Transaktion indem er sie an der richtigen Stelle der Headerkette einordnet und dann den passenden Merkel-Branch von einem seiner Nachbarknoten anfragt. Durch diese Zusätzlichen Daten kann er nun wie in Abbildung \ref{fig:spv_chain}\citep{ethereum_white_paper} gezeigt nachprüfen ob der Hash der Transaktion wirklich in den Wurzelknoten des Merkel-Trees mit eingegangen ist.

\begin{figure}[H]
\centering
\includegraphics[width=1\linewidth]{Figures/umsetzung_btc/spv_chain}
\decoRule
\caption{Blockheader Kette}
\label{fig:spv_chain}
\end{figure}

Für den Bitcoin Teil dieser Ausarbeitung ist die Integration mit dem Peer-to-Peer Netzwerk mit Hilfe der in Java geschriebenen BitcoinJ\cite{bitcoinj} Bibliothek umgesetzt.

\subsection{Überblick}

Abbildung \ref{fig:anwendung_aufbau} skizziert die Komponenten der Glücksspielanwendung und wie diese mit ihrer Umgebung kommunizieren. Es gibt zum einen den Server auf dem die Glücksspielanwendung läuft, die Spieler und das Bitcoin Netzwerk.

\begin{figure}[H]
\centering
\includegraphics[width=1\linewidth]{Figures/umsetzung_btc/anwendung_aufbau}
\decoRule
\caption{Glücksspielanwendung Aufbau und Interaktion}
\label{fig:anwendung_aufbau}
\end{figure}

\subsubsection{Glücksspielanwedung}
\begin{itemize}
\item Server: Die Glücksspielanwendung läuft auf einem Server, der über eine Java Virtual Machine (JVM) Laufzeitumgebung verfügt, eine MySQL Datenbank und ein gewöhnliches Dateisystem besitzt.
\item Java Virtual Machine (JVM): Innerhalb der JVM läuft ein sogenannter Application Server, der eine Webanwendung nach außen bereitstellt. Auf diese Webanwendung können die Spieler über das HTTP Protokoll mittels ihres Browsers zugreifen. Die Webanwendung besteht aus mehreren Komponenten.
\item Application Server: Dieser stellt die Applikation bereit. Bei der Umsetzung der Glücksspielanwendung wurde der Open Source Application Server Wildfly\footnote{\url{http://wildfly.org/}} von Red Hat verwendet.
\item GUI: Die Weboberfläche stellt die zentrale Schnittstelle zwischen der Anwendung und dem Spieler da. Diese ist mithilfe des Tapestry\footnote{\url{http://tapestry.apache.org/}} Webframeworks von Apache umgesetzt. Detaillierte Informationen findet man in \citep{tapestry}.
\item Geschäftslogik: Diese Behandelt sowohl die vom Benutzer über die GUI ausgelösten, als auch die vom Bitcoin Netzwerk ausgelösten Events.
\item BitcoinJ: Die Java Bibliothek die zur Kommunikation mit dem Bitcoin Netzwerk verwendet wird.
\end{itemize}

\subsubsection{Spieler}
Die Spieler verfügen über einen Browser und über einen Bitcoin Client.
Mit dem Internetbrowser interagieren sie mit Glücksspielanwendung. Mit dem Bitcoin Client erstellen und empfangen Sie Zahlungen.
\subsubsection{Bitcoin Peer-to-Peer Netzwerk}
Das Peer-to-Peer Netzwerk besteht aus den anderen Teilnehmern des Netzwerks. Dies sind Full-, Light Nodes und Miner. Bei Kryptowährungsnetzwerken unterscheidet man in der Regel zwischen dem Test und Hauptnetzwerk. Den Bitcoins des Testnetzwerks wird kein monetärer Wert zugeschrieben. Das Testnetzwerk dient dazu Software die dem Bitcoin Netzwerk interagieren soll zu testen. Möchte ein Händler Bitcoin in seinen Onlineshop integrieren, kann er so seine Implementierung testen ohne ein finanzielles Risiko einzugehen. 

\subsection{Datenmodel}
Die Grundklasse die die Anwendung verwendet ist die Klasse Pot. Diese repräsentiert ein Spiel und speichert alle relevanten Daten. Sie besteht aus einer Liste von Teilnehmern (Participants). Jeder Teilnehmer hat wie im Konzept beschrieben eine Ein- und Auszahlungsadresse.
\begin{figure}[H]
\centering
\includegraphics[width=1\linewidth]{Figures/umsetzung_btc/btc_datenmodell}
\decoRule
\caption{Java Datenmodel Klassendiagramm}
\label{fig:btc_datenmodell}
\end{figure}


\subsection{Geschäftslogik}

Die Java Klassen der Glücksspielanwendung können, wie in Abbildung \ref{fig:btc_businesslogic} gezeigt, in 3 verschiedene Gruppen unterteilt werden. Das \textbf{Core Module} enthält die Klassen des Datenmodells und ein Interface mit dem die GUI Anwendung interagiert. Das Interface entkoppelt die Anzeigelogik der GUI von der Geschäftslogik der jeweiligen Kryptowährung. Die GUI Komponente bekommt von der Schnittstelle allgemeine Daten und kümmert sich nur um deren Anzeige.
Das \textbf{Bitcoin Module} enthält die gesamte kryptowährungsspezifische Geschäftslogik. \textbf{BitcoinJ} enthält alle Klassen und Interfaces die benötigt werden um mit dem Bitcoin Netzwerk zu interagieren und Daten aus der Blockchain auszulesen.

\begin{figure}[H]
\centering
\includegraphics[width=1\linewidth]{Figures/umsetzung_btc/btc_businesslogic_pdf}
\decoRule
\caption{Java Geschäftslogik Klassendiagramm}
\label{fig:btc_businesslogic}
\end{figure}

\subsubsection{Start der Anwendung}
Beim Kompilieren der Anwendung legt man über einen Konfigurationseintrag fest, ob die Anwendung mit dem Bitcoin Haupt- oder Testnetzwerk interagieren möchte.
Für das Hauptnetzwerk wird die Java Klasse \textbf{BtcMainEjb} verwendet. Für das Testnetwerk wird die Klasse \textbf{BtcTestEjb} verwendet. Beide Klassen verwenden die gleiche Implementierung der abstrakten Oberklasse \code{AbstractBitcoinService}. Diese wiederum implementiert das von der GUI Komponente verwendete Interface \code{CryptoNetworkservice}.

\begin{lstlisting}[basicstyle=\small] %or \tiny or \small or \footnotesize etc.
import javax.ejb.Singleton;
import javax.ejb.Startup;
import org.bitcoinj.params.AbstractBitcoinNetParams;
import org.bitcoinj.params.MainNetParams;
import com.ossel.gamble.bitcoin.services.AbstractBitcoinService;
import com.ossel.gamble.core.data.enums.CryptoNetwork;

@Startup
@Singleton
public class BtcMainEjb extends AbstractBitcoinService {
    @Override
    public CryptoNetwork getCryptoNetwork() {
        return CryptoNetwork.BTC_MAIN;
    }
    @Override
    public AbstractBitcoinNetParams getNetworkParams() {
        return MainNetParams.get();
    }
}
\end{lstlisting}


Die beiden Klassen \code{BtcMainEjb} und \code{BtcTestEjb} sind mit \code{@Startup} und \code{@Singleton} annotiert. Es handelt sich um sogenannte \textbf{E}nterprise \textbf{J}ava \textbf{B}eans. Dies bedeutet, dass der Applikationsserver diese eigenständig managt und genau eine Instanz der Klasse beim Starten der Applikation erzeugt. Beim Start wird dann die mit \code{@PostConstruct} annotierte \code{startSpvNode} Methode aufgerufen und abgearbeitet. Diese konfiguriert und startet das \code{WalletAppKit}, welches die zentrale Klasse zur Interaktion mit der BitcoinJ Bibliothek darstellt.

\begin{lstlisting}[basicstyle=\small]
@PostConstruct
private void startSpvNode() {
    log.info("#### Start Bitcoin SPV Node  ####");
    currentPot = new Pot(2, 100000L);
    File walletDir = CoreUtil.getWalletDirectory();
    NetworkParameters params = getNetworkParams();
    String fileName = "bitcoin-" + params.getPaymentProtocolId();
    bitcoinAppKit = new WalletAppKit(params, walletDir, fileName) {
        @Override
        protected void onSetupCompleted() {
            log.info("#### Bitcoin SPV Node started ####");
        }
    };
    bitcoinAppKit.startAsync();
    waitUntilStarted(bitcoinAppKit);
    newBlockListener = new NewBlockListener(this);
    bitcoinAppKit.chain().addNewBestBlockListener(newBlockListener);
    coinReceivedListener = new CoinsReceivedListener(this);
    bitcoinAppKit.wallet().addCoinsReceivedEventListener(coinReceivedListener);
}
\end{lstlisting}

 In Zeile 4 wird zunächst ein neuer leerer Topf mit 2 Teilnehmern erzeugt. Anschließend wird das \code{WalletAppKit} erzeugt. Dazu bekommt dieses die gewünschten Netzwerkparameter und den Pfad zum Dateisystem in dem \code{BitcoinJ} die Blockchain- und Wallet-Daten speichern soll. Zeile 13 startet den durch das \code{WalletAppKit} repräsentierten SPV Node. Anschließend wird dem \code{WalletAppKit} noch ein \code{NewBlockListener} und ein \code{CoinsReceivedListener} hinzugefügt, um auf neue Blöcke und eingehende Zahlungen zu reagieren.

\subsubsection{GUI Events}

Der folgende Code zeigt die Implementierung der Methoden die von der GUI Komponente aufgerufen werden können.
\begin{lstlisting}
public Pot getCurrentPot() {
    return currentPot.clone();
}
public String getDepositAddress() {
    return bitcoinAppKit.wallet().freshAddress(KeyPurpose.RECEIVE_FUNDS).toString();
}
public String getCurrentBlockHash() {
    return bitcoinAppKit.chain().getChainHead().getHeader().getHash().toString();
}
public int getCurrentBlockHeight() {
    return bitcoinAppKit.chain().getChainHead().getHeight();
}
\end{lstlisting}


Zeile 2 gibt eine Kopie der Topdaten an die GUI Komponente weiter. Da es sich lediglich um eine Kopie des Objektes handelt, ist es ausgeschlossen, dass durch die GUI Komponente der Status des Topfs manipuliert werden kann. Zeile 6 erzeugt eine neue Empfangsadresse. Zeile 9 gibt den Blockhash des neusten Blocks der SPV Blockheader Kette zurück. Zeile 12 gibt die Blocknummer des neusten Blocks der SPV Blockheader Kette zurück.

\subsubsection{Bitcoin Netzwerk Events}

Immer wenn der SVP Node eine Transaktion auf eine vorher mittels der \code{bitcoinAppKit.wallet().freshAddress()} erzeugten Adresse empfängt, wird die \code{onCoinsReceived} Methode aufgerufen.

\begin{lstlisting}[basicstyle=\small]
@Override
public void onCoinsReceived(Wallet wallet, Transaction txn, Coin prevBalance, Coin newBalance) {
  log.debug("Transaction details: " + txn.toString());
  Coin value = txn.getValueSentToMe(wallet);
  Pot currentPot = service.getCurrentPot();
  if (currentPot.getExpectedBettingAmount() > value.getValue()) {
    log.warn("Player did not pay enough.");
    return;
  }
  List<Participant> participants = service.getPossibleParticipants();
  NetworkParameters params = service.getAppKit().params();
  for (TransactionOutput txnOutput : txn.getOutputs()) {
    Address a = txnOutput.getAddressFromP2PKHScript(params);
    String address = a.toString();
    for (Participant participant : participants) {
      String depositAddress = participant.getDepositAddress();
      if (depositAddress.equals(address)) {
        log.info("Received " + value.toFriendlyString() + " coins from " + participant.toString());
        participant.setReceivedAmount(value.getValue());
        String fromAddress = txn.getInput(0).getFromAddress().toString();
        participant.setPayoutAddress(fromAddress);
        wallet.addTransactionConfidenceEventListener(new TxnConfidenceListener(service, txn, participant));
      }
    }
  }
}
\end{lstlisting}




Immer wenn der SVP Node einen neuen besten Block findet, den er vorne an die Header Kette anhängen kann, wird die \code{notifyNewBestBlock} Methode aufgerufen.
\begin{lstlisting}[basicstyle=\small]
@Override
public void notifyNewBestBlock(StoredBlock block) throws VerificationException {
    log.info("New Block height = " + block.getHeight() + " hash = "
            + block.getHeader().getHash().toString());
    List<Pot> unfinishedPots = getUnfinishedPots(service.getClosedPots());
    for (Pot pot : unfinishedPots) {
        long potId = pot.getCreateTime().getTime();
        int payoutBlockHeight = pot.getPayoutBlockHeight();
        if (payoutBlockHeight > block.getHeight()) {
            log.info("Pot[" + potId + "] can not be handled yet.");
        } else if (payoutBlockHeight == block.getHeight()) {
            log.info("Pot[" + potId + "] select temorary Winner.");
            Block tmpPayoutBlock = new ExtendedBlock(block.getHeader().getHash().toString());
            pot.setPayoutBlock(tmpPayoutBlock);
            selectWinner(pot, tmpPayoutBlock);
        } else {
            log.info("Pot[" + potId + "] select final winner.");
            try {
               StoredBlock correctBlock = getPastBlock(payoutBlockHeight, block);
               String payoutBlockHash = correctBlock.getHeader().getHash().toString();
               Block finalPayoutBlock = new ExtendedBlock(payoutBlockHash);
               pot.setPayoutBlock(finalPayoutBlock);
               Participant winner = selectWinner(pot, finalPayoutBlock);
               log.info(winner.getDepositAddress() + " wins pot[" + potId + "].");
               startPayoutThread(pot);
            } catch (BlockStoreException e) {
               log.info("Couldn't select final winner of Pot[" + potId + "]:" + e.getMessage(), e);
            }
        }
    }
}
\end{lstlisting}



Diese Methode iteriert über jeden bereits geschlossenen Topf für den noch keine Auszahlung stattgefunden hat und unterscheidet 3 Fälle:
\begin{enumerate}
\item Der neue Block hat eine Blocknummer, die kleiner ist als die Blocknummer, die den Top entscheidet. In diesem Fall passiert nichts.
\item Der neue Block entscheidet den Topf, da die Blocknummer des Blocks gleich der PayoutHeigth des Topfs ist. Der Gewinner des Topfs wird selektiert, es findet allerdings keine Auszahlung statt. Die finale Auszahlung findet aus Sicherheitsgründen erst im nächsten Fall statt.
\item In diesem Fall gibt es mindestens einen Block, der auf dem Payout-Block des Topfs aufbaut. Ab diesem Zeitpunkt betrachtet die Anwendung den Gewinner als final.\footnote{An dieser Stelle kann man natürlich auch aus Sicherheitsgründen noch mehrere Blöcke abwarten, bevor die Anwendung eine Auszahlung startet.} Daher wird der Gewinner des Topfs überschrieben und die Auszahlung in einem neuen \code{Thread} gestartet.
\end{enumerate}

Die Klasse \code{ExtendedBlock} teilt den Blockhash zur Anzeige in der GUI in die Werte perfix, lastDigit und suffix. Die Variable lastDigit speichert die letzte numerische Stelle des Blockhashs und wird zur Gewinnerauswahl verwendet.

\begin{lstlisting}
public class ExtendedBlock extends Block {

    private String prefix;
    private String suffix;

    public ExtendedBlock(String blockHash) {
        super(blockHash, -1);
        int position = blockHash.length() - 1;
        while (position > 0) {
            char c = blockHash.charAt(position);
            int value = (int) c;
            if (value >= 48 && value <= 57) {// numeric
                this.lastDigit = Integer.parseInt(String.valueOf(c));
                break;
            }
            position--;
        }
        this.prefix = blockHash.substring(0, position);
        this.suffix = blockHash.substring(position + 1, blockHash.length());
    }
}
\end{lstlisting}







\subsubsection{Auszahlungen}
Auszahlung werden in einem eigenen Thread abgehandelt. Die Klasse \code{PayoutThread} ruft dazu die \code{payout} Methode auf.
\begin{lstlisting}[basicstyle=\small]
private void payout(Pot pot) throws InsufficientMoneyException, InterruptedException, ExecutionException {
    if (pot.isPayoutStarted()) {
        log.error("Payout already started: " + pot.getPayoutTxnId() + " - " + pot.getPayoutError());
    } else {
        pot.setPayoutStarted(true);
        Address winnerAddress = new Address(bitcoinService.getNetworkParams(), pot.getWinner().getPayoutAddress());
        Coin potValue = Coin.SATOSHI.multiply(pot.getParticipants().size() * pot.getExpectedBettingamount());
        Wallet.SendResult result = bitcoinService.getAppKit().wallet()
        .sendCoins(bitcoinService.getAppKit().peerGroup(), winnerAddress, potValue);
        String txnId = result.tx.getHash().toString();
        pot.setPayoutTxnId(txnId);
        log.info("Payout TXN ID = " + txnId);
        Transaction transaction = result.broadcastComplete.get();
        log(transaction);
    }
}
\end{lstlisting}


Die Methode prüft erst, dass die Auszahlung noch nicht gestartet wurde. Ist dies der Fall, wird der auszuzahlende Betrag berechnet und an die Adresse des Gewinners überwiesen. Anschließen wird die ID der Auszahlungstransaktion in den Topf geschrieben. Während der Auszahlung können von BitcoinJ 3 verschiedene Fehler auftreten:
\begin{enumerate}
\item InsufficientMoneyException: Falls die von der Wallet verwalteten Adressen nicht genug Bitcoin für die Auszahlung besitzen.
\item InterruptedException: Falls der Java Thread unterbrochen wird.
\item ExecutionException: Falls es zu einem unerwarteten Fehler bei der Ausführung kommt.
\end{enumerate}
Sollte es bei der Auszahlung ein Problem geben, wird die Exception gefangen und abgespeichert.

\subsection{Grafische Benutzeroberfläche}\label{ssec:btc_gui}

Die Graphische Oberfläche der Anwendung ist mit dem Tapestry Framework von Apache realisiert. Da die GUI Komponente nur die Daten visualisiert, wird an dieser Stelle auf eine genauere Betrachtung verzichtet und lediglich die Benutzeroberfläche gezeigt.


\begin{figure}[H]
\centering
\includegraphics[width=1\linewidth]{Figures/btc_gui/pot_open2}
\decoRule
\caption{Leerer Topf}
\label{fig:pot_open2}
\end{figure}
Abbildung \ref{fig:pot_open2} zeigt einen Topf mit 2 freien Plätzen. Um dem Spiel beizutreten, muss der Spieler den Betrag von 0,001 Bitcoin an die angezeigte Adresse senden. Der angezeigte QR-Code erleichtert dem Spieler die Übertragung dieser Daten in das Überweisungsformular seines Smartphone Wallets. Das \textbf{B}itcoin \textbf{I}mprovement \textbf{P}roposal Nummer 21\citep{bip21} legt fest, in welchem Format diese Daten kodiert werden müssen, damit beliebige Bitcoin Clients diese korrekt auslesen können. 
Folgende Daten sind in dem QR Code enthalten: ''bitcoin:mws1YS...Ae5?amount=0.01\&message=gambling''

\begin{figure}[H]
\centering
\includegraphics[width=1\linewidth]{Figures/btc_gui/handy_send}
\decoRule
\caption{Zahlung mit Smartphone}
\label{fig:handy_send}
\end{figure}
Abbildung \ref{fig:handy_send} zeigt das Bitcoin CoPay Wallet\footnote{\url{https://copay.io/}}. Dieses erlaubt es Zahlungen an das Bitcoin Testnetz zu senden. Nachdem der Benutzer den QR-Code der Glücksspielanwendung mit seinem Smartphone abgescannt hat, erscheint sowohl der Betrag als auch die Empfangsadresse vor-ausgefüllt im Überweisungsformular. Die Wallet berechnet automatisch eine passende Transaktionsgebühr. Der Spieler prüft lediglich die Adresse und den Betrag und  autorisiert anschließend die Zahlung.

\begin{figure}[H]
\centering
\includegraphics[width=1\linewidth]{Figures/btc_gui/pot_closed}
\decoRule
\caption{Topf geschlossen}
\label{fig:pot_closed}
\end{figure}
Nachdem die Zahlung von der Glücksspielanwendung erkannt wurde und ein weiterer Spieler in den Topf eingezahlt hat, ist der Topf voll. Nun wartet die Anwendung auf den Payout Block, der den Gewinner des Topfs festlegt. Da der Blockhash noch nicht feststeht, zeigt die Anwendung die Animation eines sich sehr schnell wechselnden Block Hash, der mehrere Fragezeichen enthält, an. Dies ist in Abbildung \ref{fig:pot_closed} zu sehen.

\begin{figure}[H]
\centering
\includegraphics[width=1\linewidth]{Figures/btc_gui/pot_settled}
\decoRule
\caption{Gewinner ermittelt}
\label{fig:pot_settled}
\end{figure}
Sobald der entscheidende Block empfangen wird, wird der voraussichtliche Gewinner des Topfs durch ein rotes Rechteck markiert. Nun wartet die Anwendung bis ein weiterer Block gefunden wird, bevor sie die Auszahlung an den Gewinner startet. 

\begin{figure}[H]
\centering
\includegraphics[width=1\linewidth]{Figures/btc_gui/pot_payout_finished}
\decoRule
\caption{Auszahlung beendet}
\label{fig:pot_settled}
\end{figure}

Klickt der Benutzer auf den \emph{Payout triggered} Link, gelangt er zu einem Blockexplorer. Dieser zeigt ihm die Transaktionsdetails an. 